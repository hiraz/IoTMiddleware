

\documentclass[journal]{IEEEtran}

% *** CITATION PACKAGES ***
\usepackage{cite}
\usepackage{float,lscape}

% *** GRAPHICS RELATED PACKAGES ***
\ifCLASSINFOpdf
 \usepackage[pdftex]{graphicx}
  % declare the path(s) where your graphic files are
  % \graphicspath{{../pdf/}{../jpeg/}} and their extensions so you won't have to
  % specify these with every instance of \includegraphics
  % \DeclareGraphicsExtensions{.pdf,.jpeg,.png}
\else
  % or other class option (dvipsone, dvipdf, if not using dvips). graphicx will
  % default to the driver specified in the system graphics.cfg if no driver is
  % specified.
  % \usepackage[dvips]{graphicx} declare the path(s) where your graphic files
  % are \graphicspath{{../eps/}} and their extensions so you won't have to
  % specify these with every instance of \includegraphics
  % \DeclareGraphicsExtensions{.eps}
\fi


% *** MATH PACKAGES ***
\usepackage[cmex10]{amsmath}
\usepackage{multirow} 
\usepackage{multicol} 
\usepackage{tabu}
% *** SPECIALIZED LIST PACKAGES ***
\usepackage{algorithmic}
% algorithmic.sty was written by Peter Williams and Rogerio Brito.
\usepackage{array}

% *** SUBFIGURE PACKAGES ***
\ifCLASSOPTIONcompsoc
  \usepackage[caption=false,font=normalsize,labelfont=sf,textfont=sf]{subfig}
\else
  \usepackage[caption=false,font=footnotesize]{subfig}
\fi

% *** FLOAT PACKAGES ***
\usepackage{fixltx2e}
% fixltx2e, the successor to the earlier fix2col.sty, was written by
% \usepackage{stfloats} stfloats.sty was written by Sigitas Tolusis. This
% package gives LaTeX2e
\usepackage{dblfloatfix}
\usepackage{color,soul}
% The latest version can be found at:
% http://www.ctan.org/tex-archive/macros/latex/contrib/dblfloatfix/

\usepackage{url}


% correct bad hyphenation here
\hyphenation{op-tical net-works semi-conduc-tor}


\begin{document}
%  paper title can use linebreaks \\ within to get better formatting as desired
% Do not put math or special symbols in the title.
\title{Middleware for IoT: state-of-the art and challenges}
%   author names and IEEE memberships note positions of commas and nonbreaking
% spaces ( ~ ) LaTeX will not break a structure at a ~ so this keeps an author's
% name from being broken across two lines.
% use \thanks{} to gain access to the first footnote area a separate \thanks
% must be used for each paragraph as LaTeX2e's \thanks was not built to handle
% multiple paragraphs

\author{M.A.~Razzaque,%~\IEEEmembership{Member,~IEEE,}
        ~ Marija ~M. Jevric,
        ~Andrei Palade,
       ~ and Siobh\'an~Clarke,
        % % stops a space
\thanks{M. A. Razzaque, Marija ~M. Jevric, Andrei Palade, and
Siobh\'an~Clarke are with the Distributed Systems Group, School of Computer
Science and Statistics, Trinity College Dublin, e-mail: razzaqum@tcd.ie}% <-this stops a space

% \thanks{, is wwith the Distributed Systems Group, School
% of Computer Science and Statistics, Trinity College Dublin, e-mail:
% razzaqum@tcd.ie }
% 
% \thanks{, is with the Distributed Systems Group,
% School of Computer Science and Statistics, Trinity College Dublin, e-mail:
% razzaqum@tcd.ie}%
% 
% \thanks{Siobh\'an~Clarke is with the Distributed Systems Group, School of
% Computer Science and Statistics, Trinity College Dublin, e-mail:
% razzaqum@tcd.ie}%
% <-this % stops a space
\thanks{Manuscript received Dec. 24, 2014; revised , 2015.}}


% The paper headers
\markboth{IEEE INTERNET OF THINGS JOURNAL,~Vol.~0, No.~0, January~201x}
{Razzaque \MakeLowercase{\textit{et al.}}: Middleware for IoT:
state-of-the art and challenges}




% make the title area
\maketitle

% As a general rule, do not put math, special symbols or citations in the
% abstract or keywords.
\begin{abstract}
The Internet-of-Things (IoT) envisages a future in which digital and physical things
or objects (e.g., smartphones, TVs, cars) can be connected by means of suitable
information and communication technologies, to enable a range of
applications and services. The IoT's characteristics, including an ultra large-scale
network of things, device and network level heterogeneity, and the large number
of events generated spontaneously by these things, will make development of the diverse
applications and services a very challenging task. In general, middleware can
ease the development process by integrating heterogeneous computing and
communications devices and supporting interoperability within the diverse
applications and services. Recently, there have been a number of proposals
for IoT middleware. These proposals are mostly based on Wireless Sensor Networks
(WSNs), a key component of IoT but do not consider Radio-Frequency
IDentification (RFID), Machine to Machine (M2M) communications, and Supervisory
Control and Data Acquisition (SCADA), other three core elements in the IoT
vision. Taking a holistic view, in this article, we outline a set of requirements for 
IoT middleware, and present a comprehensive review of the existing
middleware solutions against those requirements. In addition, open research issues, challenges and future research
directions are highlighted.


\end{abstract}
% Note that keywords are not normally used for peerreview papers.
\begin{IEEEkeywords}
Mobile Adhoc Networks, Vehicular Adhoc Networks, Adaptive Composition, Dynamically Adaptive, Scalability
\end{IEEEkeywords}



% For peer review papers, you can put extra information on the cover
% page as needed:
% \ifCLASSOPTIONpeerreview
% \begin{center} \b
% For peerreview papers, this IEEEtran command inserts a page break and
% creates the second title. It will be ignored for other modes.
\IEEEpeerreviewmaketitle


\section{Introduction}



With the advance of numerous technologies including sensors, actuators,
embedded computing and cloud computing, and the emergence of a new generation of
cheaper, smaller wireless devices, many objects or things in our
daily lives are becoming wirelessly interoperable with attached miniature and
low-powered or passive wireless devices (e.g., passive RFID
tags). The Wireless World Research Forum predicts that by 2017, there will be 7
trillion wireless devices serving 7 billion people~\cite{WWRF2006}, one thousand
devices per person. This ultra large number of connected things or devices will
form the Internet of Things (IoT)~\cite{FutureInternet07, IoTCharacterstics10}.

By enabling easy access of, and interaction with, a wide variety of physical devices
or things such as, home appliances, surveillance cameras, monitoring sensors,
actuators, displays, vehicles, machines and so on, the IoT will foster the
development of applications in many different domains, such as home automation,
industrial automation, medical aids, mobile healthcare, elderly assistance,
intelligent energy management and smart grids, automotive, traffic management,
and many others~\cite{IoTSmartCity13}. These applications will make use of the
potentially enormous amount and variety of data generated by such objects to
provide new services to citizens, companies, and public
administrations~\cite{IoTvison13,IoTCharacterstics10}.


In a ubiquitous computing environment like IoT, it is impractical to impose
standards and make everyone comply. An ultra large-scale network of things and the
large number of events that can be generated spontaneously by these things, along
with heterogeneous devices/technologies/applications of IoT bring new
challenges in developing applications, and make the existing challenges in 
ubiquitous computing considerably more
difficult~\cite{FutureInternet07,IoTCharacterstics10}. In this context, a
middleware can offer common services for applications and ease application
development by integrating heterogeneous computing and communications devices
and supporting interoperability within the diverse applications and services
running on these devices. Complementary to middleware are programming language
approaches~\cite{SemanticMiddleware08,MANETProgarmming05}. These approaches 
tackle some of the challenges (such as discovery, network disconnections, and
group communication) posed by the IoT, fail to support others such as
context-awareness (e.g., context-aware service discovery) and scalability.



WSNs, RFID, M2M communications, and SCADA are the four essential components of IoT~\cite{IoTMiddlewareBook12,
ContextIoT13}. A fully functional IoT middleware needs to integrate WSNs, RFID,
M2M, and SCADA technologies to support the envisioned diverse application
domains~\cite{IoTMiddlewareBook12}. Unfortunately, most existing proposals and
surveys~\cite{GSN06,azzara2013middleware,issarny2011service,
SurveyIoTmiddleware11,Book2013} for IoT middleware do not consider these technologies in a single solution. 
Moreover, the majority of the existing IoT
middleware proposals~\cite{aberer2006global, musolesi2005emma,fok2009agilla,
SurveyIoTmiddleware11,silva2014prisma} are WSNs centric. A number
of surveys have been conducted on WSNs middlewares~\cite{hadim2006middleware,
WangSurvey2008,schmidt2002middleware, MiddlewareWSNSurvey12,
li2014middleware,bhuyan2014survey}, these are either not comprehensive ~\cite{MiddlewareWSNSurvey12,
li2014middleware,bhuyan2014survey} or do not report more 
recent work ~\cite{hadim2006middleware, WangSurvey2008,schmidt2002middleware}. 
From these surveys, it is evident
that no single existing middleware can support all the necessary functional and
non-functional requirements for WSNs as well as IoT applications. For instance,
Perera et al \cite{ContextIoT13} identified that most existing WSN
middleware and IoT-focused solutions do not support context-awareness. In addition, 
unlike WSNs, the number of middleware proposals for RFID as well as M2M
communications and SCADA is limited~\cite{RFIDMiddleware09, scada10,
M2M11,IoTMiddlewareBook12, HeterogenistyIoTMiddleware12, M2M13}.

 Research into IoT, especially in IoT middleware is still in its early stage.
 Nonetheless, IoT-specific middlewares are emerging ~\cite{ IoTMiddleware10, SematnicMiddlewareIoT10
 ,ResourceMiddlewareIoT12, IoTMiddlewareBook12,SoAIoTMiddleware11,Book2013,
 MOSDEN14} as are some surverys~\cite{RoleofMiddlewareIoT11,SurveyIoTmiddleware11,IoTMiddlewareBook12}. 
 Bandyopadhyay et al ~\cite{RoleofMiddlewareIoT11} have highlighted
 the importance of a middleware system in IoT. They also presented a survey
 on IoT middlewares in \cite{SurveyIoTmiddleware11}. However, this is already
dated, and does not include most IoT-specific
 middlewares~\cite{ResourceMiddlewareIoT12,
 IoTMiddlewareBook12,SoAIoTMiddleware11,Book2013, MOSDEN14}.
 Zhou has presented an overview of the existing
 middlewares for WSNs, RFID, M2M and SCADA  ~\cite{IoTMiddlewareBook12}, 
 and has also presented a unified
 framework for IoT middleware based on service orientation. Implementation
 detail is not available and like~\cite{SurveyIoTmiddleware11}, 
 \cite{IoTMiddlewareBook12}, does not include recent, and IoT-specific
 middlewares~\cite{Book2013, MOSDEN14}.


Considering the importance of IoT in various domains, this article 
takes a holistic view of middleware for IoT and (i)
identifies the key characteristics of IoT and the requirements of IoT's
middleware (section 2), (ii) based on the identified requirements, presents a
comprehensive review of the existing middleware systems focusing on current,
state-of-the-art research (section 3),  and (iii) outlines open
research challenges, recommending future research directions (section 4).


\section{Background}

\subsection{IoT and its Characteristics}
%
In recent years, the IoT has gained significant attention in academia
and industry~\cite{IoTSurvey10}. IoT enables a world where all the objects
around us will be connected to the Internet and interact with each other with very
little or no human intervention~\cite{SemanticWeb09}. The eventual goal is to
make a better world for humans, where things or objects around us know what we like, what
we want, and what we need and act accordingly without our explicit
instructions~\cite{AssistedLiving10}.



Research into IoT is still in its early stage, and a standard definition of
IoT is not yet available. IoT can be viewed from three
perspectives:
Internet-oriented, things-oriented (sensors or smart things) and
semantic-oriented (knowledge)~\cite{IoTSurvey10}. Also, the IoT can be viewed
as either supporting consumers (human) or industrial applications and indeed could
be named as the Human Internet of Things (HIoT) or the Industrial Internet of
Things (IIoT)~\cite{IIoT13a,IIoT14,IIoT14a,IoTMiddlewareBook12}. Even though these different
views have evolved because of the interdisciplinary nature of the subject, they are likely
to intersect in an application domain to achieve the goals of IoT.



The first definition of IoT was from a ``things oriented"
perspective, where RFID tags were
considered as things~\cite{IoTSurvey10}. According to the RFID community, IoT
can be defined as~\cite{IoT08}, ``The worldwide network of interconnected objects uniquely
addressable based on standard communication protocols''. According to the
European Research Cluster of IoT~\cite{IoT10,IoT13}, ``The Internet of Things
allows people and things to be connected Anytime, Anyplace, with Anything and
Anyone, ideally using Any path/network and Any service.'' The International
Telecommunication Union (ITU) views IoT very similarly 
~\cite{ITU05}: ``From anytime, anyplace connectivity for anyone, we
will now have connectivity for anything". Semantically, 
IoT means ``a world-wide network of interconnected objects uniquely
addressable, based on standard communication protocols"~\cite{IoT08}.

Most existing definitions of IoT do not explicitly highlight the industrial view of
IoT (IIoT). World leading companies are giving special attention and making
 significant investments in the IoT for their industrial solutions (IIoT).
 Even though they are using different and catchy terms such as ``Smarter
 Planet'' by IBM, ``Internet of Everything'' by Cisco and ``Industrial
 Internet'' by GE, their main objective is to use IoT to improve industrial
 production by reducing unplanned machine downtime and significantly reducing
 energy costs along with number of other potential benefits~\cite{IIoT13a,
 IIoT13b,IIoT14,IIoT14a,IoTMiddlewareBook12}. The IIoT refers to industrial
 objects, or ``things'', instrumented with sensors, automatically communicating
 over a network, without human-to-human or human-to-computer
interaction, to exchange information and take intelligent decisions with the
support of advanced analytics~\cite{IIoT14a}.



The definition of ``things'' in the IoT vision is very wide and includes a
variety of physical elements.
These include personal objects we carry around such as smart phones, tablets and
digital cameras. It also includes elements in our environments (e.g. home,
vehicle or work), industries (e.g., machines, motor, robot) as well as things
fitted with tags (RFID or other) which become connected via a gateway device (e.g., a smart
phone). Based on this view of ``things'', an enormous number of devices and
things will be connected to the Internet, each providing data and information
and some, even services.


Sensor Networks (SNs) including wireless sensor networks (WSNs), RFID, M2M
communications and Supervisory
Control and Data Acquisition (SCADA) are the
essential components of IoT. As described in more detail in this section, a number 
of the IoT's characteristics are inherited from one or more of
these components. For instance, resource
constrained is inherited from RFID and SNs, and intelligence is inherited from
WSNs and M2M.  Other characteristics (e.g., ultra large scale network,
spontaneous interactions) are specific to the IoT. In the following, we briefly 
present the main characteristics of
the IoT from infrastructure and application perspectives. 

\begin{figure}[ht!]
\centering
\includegraphics[width=.65\columnwidth]{figures/iot.pdf}
\caption{Definition of IoT~\cite{IoT10}.}
\label{fig4}
\end{figure}


\subsubsection{Characteristics of IoT Infrastructure}



\begin{itemize}

\item[$\bullet$] \textit {Heterogeneous Devices:} The embedded and sensor
computing nature of many IoT devices means that low-cost computing platforms are likely to be used.
In fact, to minimize the impact of such devices on the environment and energy
consumption, low-power radios are likely to be used for connection to the
Internet.
Such low-power radios do not use WiFi, or well established cellular network
technologies. However, the IoT will not be composed only of embedded devices and
sensors, it will also need higher-order computing devices to perform heavier
duty tasks (routing, switching, data processing, etc.). Figure~\ref{fig1} broadly categorizes
the devices we consider as part of the IoT into 6 levels.
Heterogeneity of these devices is not only in terms of capacity and ability, it
can also be due to other reasons including multivendor products, application
requirements, etc.~\cite{ITU05,IoTSmartCity13}.

%Figure~\ref{fig1} presents examples of heterogeneous devices will be used in IoT
%,keepaspectratio
\begin{figure*}[ht!]
\centering
\includegraphics[width=1.5\columnwidth]{figures/IoTDevices.pdf}
\caption{Examples of Device Heterogeneity in IoT.}
\label{fig1}
\end{figure*}


\item[$\bullet$] \textit {Resource Constrained:} Use of embedded computing and
sensors or WSNs will need a small form factor for the devices, which limits their
processing, memory, and communication capacity of the devices used in IoT.
As shown in Figure~\ref{fig1}, resource capacity decreases (e.g.,
computational, connectivity capabilities, memory requirements) as we move from
left to right. For example, RFID devices or tags (in the right most side of the
figure) may not have any processing capacity or even no battery to power
them. On the other hand, in Figure~\ref{fig1} devices become expensive and
larger in form-factor when moving to the left.

\item[$\bullet$] \textit {Spontaneous Interaction:} 
In IoT applications, sudden interactions can take place as objects
or things move around and come into other objects' communication range, leading to
the spontaneous generation of events. For instance, a smartphone user can come
in close contact with a TV/fridge/washing machine at home and that can
generate events without the user's involvement. Typically, in IoT, an interaction
with an object means that an event is generated and is pushed to the system
without much human attention.


%% Network charaterstics

\item[$\bullet$] \textit{Ultra Large-Scale Network and Large Number of Events:}
In an IoT environment, thousands of devices or things may interact with each
other even in one local place (e.g., in a building, supermarket, university),
which is much larger scale than most conventional networking systems. Globally,
the IoT will be an ultra large scale network containing nodes in the scale of
billions and even in trillions.
Gartner has reported~\cite{Garner13} that there will be nearly 26 billion
devices on the IoT by 2020. Similarly, ABI Research~\cite{ABI13} estimated that
more than 30 billion devices will be wirelessly connected (Internet of
Everything) by 2020.

In the IoT, spontaneous interactions amongst an ultra large number of things or
devices, will produce an enormous number of events as normal behavior. This uncontrolled
number of events may cause problems such as event congestion and reduced event
processing capability.



\item[$\bullet$] \textit {Dynamic Network and No Infrastructure:} As shown in
Figure \ref{fig1}, IoT will integrate devices, most of which will be mobile,
wirelessly connected, and resource constrained.
Many nodes within the network may be mobile, and can leave or join anytime they
want. Also, nodes can be disconnected due to poor wireless links or battery
shortage. All of these factors will make the network in IoT highly dynamic.

Within such an ad hoc environment, where there is limited or no connection to a 
fixed infrastructure, it will be difficult to maintain a stable network to
support many application scenarios that depend on the IoT. Nodes or things will
need to cooperate to keep the network connected and active.
 
 \item[$\bullet$] \textit {Context-aware:} Context is key in the IoT and its
 applications. A large number of sensors will generate large amounts of data, which
 not have any value unless we analyze, interpret, and
 understand it. Context-aware computing stores context
 information related to sensor data, easing its interpretation. 
Context-awareness (especially in temporal and spatial
 context) plays a vital
role in the adaptive and autonomous behavior of the things in the
IoT~\cite{ContextIoT13,ContextIoT13b}. Such behavior will help to eliminate 
human-centric mediation in the IoT, which ultimately makes it easier to
perform machine-to-machine communication, a core element of the IoT's vision.




\item [$\bullet$] \textit{Intelligence:}  According to the Intel's IoT vision,
intelligent device or things and intelligent system of systems are the two key
elements of IoT~\cite{IoTIntel}. In IoT's dynamic and open network, these
intelligent entities along with other entities such as web services,
Service-Oriented Architecture (SOA) components, and virtual objects will be
interoperable and able to act independently based on the context, circumstances
or environments~\cite{IntillegenceIoT09,IntillegenceIoT13}.



\item[$\bullet$] \textit {Location-aware:} Location or spatial information
about things (objects) or sensors in IoT is critical, as location plays a
vital role in context-aware computing. In a large-scale network of things,
interactions are highly dependent on their locations, their surroundings, and
presence of other entities (e.g., things and people). 

\item[$\bullet$] \textit {Distributed:} The traditional Internet itself is
a globally distributed network, and so also is the IoT. The strong spatial
dimension within the IoT makes the network IoT distributed at different
scales (i.e., both globally like the Internet, and also locally within 
an application area).

\end{itemize}

\subsubsection{Characteristics of IoT Applications}

\begin{itemize}


\item[$\bullet$] \textit {Diverse Applications:} The IoT can offer its services to 
a large number of applications in numerous domains and environments. These domains and environments can be grouped, non-exhaustively, 
into domain categories such as: (i) Transportation and logistics, (ii) Healthcare,
(iii) Smart environment (home, office, plant), (iv) Industrial and (v)
Personal and social domain. These domains and environments are not exhaustive list of the
IoT applications. Figure~\ref{fig2} highlights some key application domains for the
IoT.

Different applications are likely to need different
deployment architectures (e.g., event driven, time driven) and have different
requirements, which have to date generally been handled using a proprietary
implementation. However, since the IoT is connected to the Internet, most of the
devices comprising IoT services will need to operate utilizing standardized
technologies.


\item[$\bullet$] \textit {Real-time:} Applications using the IoT can be broadly
classified as real-time and non-real-time. For instance, IoT for healthcare, transportation, etc. will need on-time delivery
of their data or service. Delayed delivery of data can make the application or
service useless and even dangerous in mission critical applications. 

\item[$\bullet$] \textit {Everything-as-a-service (XaaS)}:
An everything-as-a-service model is very efficient, scalable, and easy to
use~\cite{XaaS11}. The XaaS model has inspired the Sensing as a Service approach
in WSNs~\cite{SaaS12,SaaS13}, and  this may inevitably lead IoT toward an
everything-as-a-service (XaaS) model. As more things get connected, the
collection of services is also likely to grow and as they become accessible online, 
they will be available for use, and re-use.


\item [$\bullet$] \textit {Increased Security Attack-surface:} While there is huge
potential for the IoT in different domains, there are also concerns for the security
of applications and networks.
The IoT needs global connectivity and accessibility, which means that anyone
can access it anytime and anyway.
This tremendously increases the attack surfaces for the IoT's applications and
networks.
The inherent complexity of the IoT further complicates the design and
deployment of efficient, interoperable, and scalable security mechanisms. 

\item [$\bullet$] \textit{Privacy Leakage:}  Using the IoT, applications may
collect information about people's daily activities. As information reflecting
users' daily activities (e.g., travel routes, buying habits, daily energy usage
and so on) is considered by many individuals as private, exposure of this
information could impact the privacy of those individuals. The use of cloud
computing makes the problem of privacy leakage even worse.
Any IoT application not compliant with privacy requirements could be prohibited
by law (e.g., in the EU~\cite{PrivacyEU}) because the violate citizens' privacy.


\end{itemize}
%\section{Heterogeneous communication and computing systems and Middleware Requirements}

%\subsection{Heterogeneous communication and computing systems}

%Figure~\ref{fig1} presents an example of heterogenous communication and computing systems that can be used in smart city applications.
%
%\begin{figure}[ht!]
%\centering
%\includegraphics[width=.8\columnwidth,keepaspectratio]{hsystems.pdf}
%\caption{An example of heterogeneous communication systems.}
%\label{fig1}
%\end{figure}



\begin{figure}[ht!]
\centering
\includegraphics[width=.9\columnwidth]{figures/IoTApps.pdf}
\caption{Potential applications of IoT~\cite{IoTFig1}.}
\label{fig2}
\end{figure}





\subsection{Middleware in IoT and its requirements}

%Middleware can be viewed in communication as well as computing perspective. In
%the communication perspective, it can be described as a software layer between
%applications, the operating system and the network communications layers, the
%purpose of which is to facilitate and coordinate some aspect of cooperative
%processing. From the computing perspective, it is defined as a layer between
%application software and system software. Generally, a middleware abstracts the
%complexities of the system or hardware, allowing the application developer to
%focus all efforts on the task to be solved without the distraction of orthogonal
%concerns at the system or hardware level~\cite{MiddlewarDefinition06}.
%Considering the communication as well as computing level heterogeneities, in
%IoT, a middleware which can support both the perspectives is necessary. A
%middleware in IoT needs to fulfil a number of requirements to become useful
%and effective. Characteristics of IoT's infrastructure and its applications
%play a vital role in identifying the requirements of IoT's middleware.
%Generally, these requirements can be viewed as the requirements for the services a middleware
%caters and the system architecture of the target middleware. In the following,
%we summarize those requirements of IoT's middleware.

Generally, a middleware abstracts the complexities of the system or hardware,
allowing the application developer to focus all his effort on the task to be
solved, without the distraction of orthogonal concerns at the system or hardware
level~\cite{MiddlewarDefinition06}.
Such complexities may be related to communication concerns or to more general
computation.
For communication, a middleware provides a software layer between applications,
the operating system and the network communications layers, the purpose of which
is to facilitate and coordinate some aspect of cooperative processing. From the
computing perspective, a middleware provides a layer between application
software and system software.
In the IoT, there is likely to be considerable heterogeneity in both the
communication technologies in use, and also the system level technologies, and a
middleware should support both perspectives as necessary. In this section, we
draw on the previously described characteristics of the IoT's infrastructure and
the applications that depend on it, to identify a set requirements for a
middleware to support the IoT. As follows, we have grouped these requirements
into two sets: first, the services such a middleware should provide, and second,
the system architecture it should support.

% 
% \begin{figure}[ht!] \centering
% \includegraphics[width=.9\columnwidth]{figures/IoTApps.pdf}
% \caption{Middleware hierachy requirements in IoT.}
% \label{fig2a}
% \end{figure}


\begin{figure*}[ht!]
\centering
\includegraphics[width=2.0\columnwidth]{figures/MDReq.pdf}
\caption{Relationships between the IoT Applications and Infrastructure and
its Middleware Requirements.}
\label{fig3}
\end{figure*}



\subsubsection{Middleware Service Requirements}

Middleware service requirements for the IoT can be categorized as both
functional and non-functional. Functional requirements capture the
services or functions (e.g., abstractions, resource management) a middleware
provides and non-functional requirements (e.g., reliability, security, availability)
capture QoS support or performance issues.

Our view of a middleware in this paper is one which provides common or generic
services to multiple different application domains. In this section, we make no
attempt to capture domain or application-specific requirements, but focus on generic 
or common functional ones, as follows:

\begin{itemize}
  
  
\item[$\bullet$] \textit {Resource Discovery:} IoT resources include
heterogeneous hardware devices (e.g., RFID tags, sensors, sensor mote,
smartphones), devices' power and memory, analogue to digital convertor
device (A/D), the communications module
available on those devices, and infrastructural or network level information
(e.g., network topology, protocols) and the services provided these devices.
Assumptions related to global and deterministic knowledge of these resources'
availability are invalid, as the IoT's infrastructure and environment is
dynamic. By necessity, human intervention for resource discovery is infeasible,
and therefore an important requirement for resource discovery is that it be
automated. Importantly, when there is no infrastructure network, every device
must announce its presence and the resources it offers. This is a different
model to the traditional distributed systems one, where resource publication,
discovery and communication are generally managed by a dedicated server.
Discovery mechanisms also need to scale well, and there should be efficient
distribution of discovery load, given the IoT's composition of
resource-constrained devices.

\item [$\bullet$] \textit{Resource Management:} 
An acceptable QoS is expected for all applications, and in an environment where resources
that impact on QoS are constrained, such as the IoT, it is important that applications are provided with
a service that manages those resources. This means that resource usage should be
monitored, resources allocated or provisioned in a fair manner, and resource
conflicts resolved. In IoT architectures, especially in service oriented or
virtualization based architectures, middleware needs to facilitate resource
(service) composition if single resource or service fails to satisfy application
needs.


\item [$\bullet$] \textit{Data Management:} Data is key in the IoT applications.
In IoT, data refers mainly to sensed data or any network infrastructure information 
of interest to applications.
An IoT middleware needs to provide data management services to applications, including
data acquisition, data processing (including pre-processing), and data storage.



\item [$\bullet$] \textit{Event Management:} There are potentially a massive number of events
generated in IoT applications, which should be managed as an integral part of an
IoT middleware. Event management transforms simple observed events into
meaningful events. It should provide real-time analysis of high-velocity data
so that downstream applications are driven by accurate, real-time information and intelligence. 


\item [$\bullet$] \textit{Code Management:} Deploying code in an IoT environment
is challenging, and should be directly supported by the middleware. In particular,
code allocation and code migration services are required. Code
allocation selects the set of devices or sensor nodes to be used to 
accomplish a user or application level task. Code
migration transfers one node/device's code to another one, potentially
reprogramming nodes in the network. Using code migration services, code is portable, 
which enables data computation to be re-located. For instance, code can be transferred 
to nodes close to 
the area where a comparatively large proportion of the data is collected, facilitating
potential energy savings through reduced communications.
\end{itemize}


Key non-functional requirements of IoT
middleware includes scalability, reliability, availability, real-time
(timeliness), security and privacy, and ease-of deployment.

\begin{itemize}

\item[$\bullet$] \textit{Scalability:} An IoT middleware is scalable when it can
accommodate an ever increasing growth in the IoT's network and
applications/services. Considering the size of the IoT's network, IPv6 is a very
scalable solution for addressability, as it can deal with a huge number of
things that need to be included in the IoT~\cite{atzori2010internet}. Loose
coupling and/or virtualization in middleware is useful in improving the
scalability, especially application and service level scalability by hiding the
complexity of the underlying hardware or service logic and implementation.




\item[$\bullet$] \textit{Real-time or Timeliness:} A middleware must provide 
real-time services when the total correctness of an operation it supports depends 
not only on its logical correctness, but also on the time in which it is performed. 
As the IoT will deal with many
real-time applications (e.g., transportation, healthcare), on-time delivery of
information or services in those applications is critical. Delayed information or
services in such applications can make the system useless and even dangerous.
%Delay  of different operations of a middleware in dynamic
%environments of the IoTs will be useful in judging the real-time or timeliness
%behavior of it.

\item[$\bullet$] \textit{Reliability:} A middleware is reliable if it can remain
operational for the duration of a mission, even in the presence of failures.
In the IoT, reliability has two aspects:
communication reliability and data or information reliability, both of which 
should be considered. Reliability is an
end-to-end QoS metric in the IoT, and technologies and devices from all layers are
responsible for it.

\item[$\bullet$] \textit{Availability:} A middleware supporting an IoT's
applications, especially mission critical ones, must 
be available, or appear available, at all times. Even if there is a failure
somewhere in the system, its
recovery time and failure frequency must be small
enough to achieve the desired availability. 
The reliability and availability requirements should work together to ensure 
the highest fault tolerance
required from an application. 


\item[$\bullet$] \textit{Security \& Privacy:} Security at both the things or
device and network levels including middleware is critical to the operation of
IoT. In IoT middleware, security needs to be considered in all the functional
and non-functional blocks including the user level application.
Context-awareness in middleware may disclose personal information (e.g. the
location of an object or a person). Like security, every block of middleware,
which uses personal information, need to preserve privacy of the data owners.


\item[$\bullet$] \textit{Ease-of deployment:} Since an IoT middleware 
(or more likely, updates to the middleware) is typically deployed by the user (or owner of the device), 
deployment should not require expert knowledge or support. 
Complicated installation and setup procedures must be avoided.

\end{itemize}


\subsubsection{Architectural Requirements}

The architectural requirements included in this section are designed to support
application developers. They include requirements for programming abstractions,
and other implementation-level concerns.

\begin{itemize}
  

\item [$\bullet$] \textit{Programming Abstraction:} Providing an API for
application developers is the foundation and most important functional
requirement for any middleware.
For the application or service developer, high-level programming interfaces
isolate the development of his applications or services from the operations
provided by the underlying, heterogeneous IoT infrastructures.
The level of abstraction, the programming paradigm, and the interface type all
need to be considered when defining an API. The level of abstraction refers to
how the application developer views the system (e.g., individual node/device
level, system level). The programming paradigm (e.g., Publish/Subscribe) deals
with the model for developing or programming the applications or services. The
interface type defines the style of the programming interface. For instance,
descriptive interfaces offer SQL-like languages for data query, XML-based
specification files for context configuration.

\item[$\bullet$] \textit{Inter-operable:} A middleware is interoperable if it
can work with heterogeneous devices/technologies/applications or even with other
middlewares without additional effort from the application or service developer.
Heterogeneous components must be able to exchange data and services.
Interoperability in a middleware can be viewed from network, syntactic, and
semantic perspectives, each of which must be catered for in an IoT.
A network should exchange information across different networks, potentially
using different communication technologies.
Syntactic interoperation should allow for heterogeneous formatting and encoding
structures of any exchanged information or service. Semantic interoperability is
related to the meaning of information or service. It can offer
interoperability for the ever-growing and changing set of devices and services
in IoT. Meaningful information about services will be useful for the users in
composing multiple services as semantic data can be better understood by
``things'' and humans compared to traditional protocol
descriptions~\cite{semanticInteroperability08,semanticInteroperability08a,SemanticIoT10,semanticInteroperability14}.


\item[$\bullet$] \textit{Service-oriented:} A service-oriented paradigm will
provide high flexibility when new and advanced functions need to be added to the IoT's
middleware. Service-oriented middleware (SOM) logically views the IoT as a service
provider for user applications. A SOM provides abstractions for the complex
underlining hardware through a set of services (e.g., data management,
reliability, security) needed by the IoT's applications. All these and other advanced
services can be designed, implemented, and integrated in a SOM framework to
deliver a flexible and easy environment for application development.


\item[$\bullet$] \textit{Adaptive:} A middleware is adaptive when it can evolve to fit
itself into changes in its environment or circumstances. In the IoT, the network and its
environment are likely to change frequently. In addition, application-level
demands or context are also likely to change frequently. To ensure user satisfaction and
effectiveness of the IoT, a middleware needs to dynamically adapt or adjust itself
to fit all such variations. 

\item[$\bullet$] \textit{Context-aware:} Context-awareness is a key requirement
in building adaptive systems and also in establishing value from sensed data.
The IoT's middleware architecture needs to be aware of the context of users,
devices, and the environment and use these for effective and essential
services' offerings to users, as outlined earlier. 

\item[$\bullet$] \textit{Autonomous:} Autonomous means self-governed.
Devices/technologies/applications are active participants in the IoT's processes
and they should be enabled to interact and communicate among themselves without
direct human intervention~\cite{gubbi2013internet,wang2010adaptive}.
Use of intelligence including \textit{autonomous agents}, embedded
intelligence~\cite{EmbeddedIntelligence11}, predictive and proactive approaches
(e.g., a prediction engine) in middleware can fulfil this
requirement~\cite{modukuri2005autonomous}.


\item[$\bullet$] \textit{Distributed:} A large-scale IoT system's
applications/devices/users (e.g., WSNs, VANETs) exchange information and
collaborate with each other. Such applications/devices/users are likely to be
geographically distributed, and so a centralized view or middleware implementation 
will not be sufficient to support many distributed services or
applications. A middleware implementation needs to support functions that are
distributed across the physical infrastructure of the IoT.


\end{itemize}
%\subsubsection{Requirements from the state-of-the-art}


Figure~\ref{fig3} presents the relationships between the IoT's middleware
requirements and its infrastructural and application characteristics. As shown in
the figure, most of the requirements are directly related (green color text)
to one or more characteristics of the IoT. A few of them are also indirectly linked
(black text) to one or more characteristics of the IoT. For instance,
the real-time behaviour requirement is directly related to the
application's real-time characteristics and indirectly to the large number of
events. Also, a few of the middleware requirements (e.g., resource
discovery, resource management) jointly capture the same set of IoT characteristics. 
%This could be due their non-orthogonal relationships.




\section{Overview of the existing works}
 
 
 
Middleware in IoT is a very active research area. A significant number of number
solutions have been proposed and implemented in this area especially in last
couple of years.
These solutions area are highly diverse in their design approaches (e.g. event
based, database), level of programming abstractions (e.g. local or node level,
global or network level), and implementation domains (e.g.
WSNs, RFID, M2M, SCADA). In IoT, a fully functional middleware should support
all the implementation domains of IoT (e.g. WSNs, RFID, M2M, SCADA). It also
should cater node or device as well as network level programming abstractions to
become distributed, and to support number of functional (e.g. resource
discovery, resource management, data management) and non-functional requirements
(e.g. security) in a comprehensive way. In this work, we group the existing
works based on their design approaches. Middlewares within each design approaches will be highlighted on the basis their supported
requirements. The design approaches of the existing middleware solutions can be
listed as below:
 
 
\begin{itemize}
 
\item Database approach
\item Event based approach
    \begin{itemize}
        \item Service oriented approach
        \item Message oriented approach
    \end{itemize}
\item Application driven approach
\item Modular approach
\item Virtual machine approach
\item Tuple space approach
\end{itemize}
 
 
\subsection{Database oriented design}
 
 
In this approach, the sensor network is considered a virtual
database system.  An application can query the database using a
Structured Query Language (SQL) like language which enables the
possibility to formulate complex
queries\cite{azzara2013middleware}. For example,  Figure
\ref{fig:sample_query} shows a sample query that calculates the
average of temperature in a certain location monitored by sensors
for a given sample size. Although this design Previous research in
this area has been focused on developing a distributed database
approach
 
 
Other properties:
\begin{enumerate}
\item Entire sensor network is abstracted as a virtual relational
database. Ease to interoperate with existing systems. \item
Doesn't support real-time applications, and provides only
approximate results
\end{enumerate}
 
 
% This is SQL script and needs to look like SQL
\begin{center}
\begin{figure}[h!]
    \begin{quote}
        SELECT AVG(temp) FROM sensor\_list  \\
        WHERE ( location.X BETWEEN 0 AND 10)  \\
        AND  (location.Y BETWEEN 0 AND 10) \\
        SAMPLE\_PERIOD 10 seconds
    \end{quote}
\caption{A sample query} \label{fig:sample_query}
\end{figure}
\end{center}
 
 
The example of database oriented middleware is \textit{SINA}. It
provides a support for both SQL-like queries and SQTL (Sensor
Query and Tasking Language). \textit{SINA} is Event-handling
middleware and it can handle problems like mobility of the
querying (sink) node \cite{henricksen2006survey}. It is unclear
whether \textit{SINA} has been implemented or tested.
 
\textit{IrisNet} (Internet-Scale Resource-Intensive Sensor Networks Services) is
database design oriented SOFTWARE platform which deploys heterogeneous services
on WMSNs. \textit{IrisNet} allows harnessing a global, wide-area sensor network
by performing Internet-like queries on this infrastructure. \textit{IrisNet}
allows users to perform Internet-like queries to video sensors and other data.
Each query operates over data collected from the global sensor network, and
allows simple Google-like queries as well as more complex queries involving
arithmetic and database operators the architecture of IrisNet is two-tiered:
heterogeneous sensors implement a common shared interface and are called sensing
agents (SA), while the data produced by sensors is stored in a distributed
database that is implemented on organizing agents (OA). Different sensing
services are run simultaneously on the architecture. \textit{IrisNet} is tested
and the results are publish in \cite{akyildiz2007survey}. It is tested for only
60 scalar sensor. It should allow testing more complex algorithms and assess the
scalability of the communication protocols under examination.
 
\textit{HyCache} a User-Level Caching Middleware for Distributed FILE Systems is
the middleware with database oriented design. It has heterogeneous storage
devices for distributed file systems and provides standard POSIX interfaces
through FUSE. It works completely in the user space. The overhead of user-level
APIs (i.e. libfuse) is negligible with multithread support on SSD, and with
appropriate tuning can even outperform the kernel-level implementation (e.g.
Seagate Momentus XT). Evaluation of the performance is conducted and published
in \cite{zhao2013hycache}.
 
The Global Sensor Networks (GSN) middleware is developed based on
usage of virtual sensors, using Peer to Peer communication between
the nodes, data stream a virtual sensor produces is encoded in
XML, and deploys SQL queries to manage data streams within the
virtual sensor. Therefore, GSN is classified as a hybrid oriented
middleware. Virtual sensor can be any device (telephone, desktop,
laptop) that accept multiple input data streams and produces
exactly one output data stream. GSN' s ability to access multiple
stream producers simultaneously. Virtual nodes have high-level
attributes like: control of the processing priority, control and
management of resources of a virtual sensor, control of the stored
data and disconnection if the amount of storage provided is
temporary disconnected.
 
Using the declarative specifications, virtual sensors can be
deployed and reconfigured in GSN containers at runtime which makes
GSN be real-time middleware. GSN is created in order to highly
dynamic process environment and allow the system to quickly react
to changing processing needs and environmental conditions. Dynamic
resource management accomplishes three main tasks: Resource
sharing, Failure management and Explicit resource control. In
\cite{aberer2006global}, the experiment was conducted in order to
show the scalability of GSN in the number of queries and clients.
It is experienced that as the number of clients increases, the
average processing time for each client decreases. GSN provides
simple and uniform access to the host of heterogeneous
technologies available and it is easy to be deployed.
 
KSpot\textsuperscript{+} is a data-centric distributed middleware
architecture for WSN \cite{andreou2011towards}. It has capability
of network-awareness and advanced query semantics to the data
acquisition by combining the following three components: Tree
Balancing Module, Workload Balancing Module and Query Processing
Module. KSpot\textsuperscript{+} is an open-source middleware
framework that can be utilised in numerous application domains
including environmental monitoring, structural monitoring, urban
monitoring, health monitoring, etc. It has been deployed in two
projects so far. By building the KSpot\textsuperscript{+}
middleware, several design enquires are taken into the
consideration, such as: Modularity, Energy-Efficiency, Distributed
and Autonomous Behavior, Scalability and Accuracy in the presence
of Failures. \textbf{Modularity} means decomposition of a system
into a number of components that may be mixed and matched in the
way to ensure high degree of openness and usability.
\textbf{Energy-Efficiency} is provided in the way that each module
is founded on the premise of energy-conscious algorithms that
minimise energy consumption and increase network longevity.
KSpot\textsuperscript{+} provides \textbf{Distributed and
Autonomous Behavior} by focusing on a fully autonomous and
decentralised behaviour of it own client-side components. More
specifically, the maintenance of any global state or data
structures is being minimised at a centralised location and use
only local knowledge. In developing of KSpot\textsuperscript{+},
special attention was given to a \textbf{Scalability} property to
ensure that the performance of the system will maintain acceptable
QoS standards regardless of the increasing network size. The
algorithms that operate in each of the KSpot\textsuperscript{+}
modules have proven to work accurately and efficiently in the
presence of failure and providing the \textbf{Accuracy in the
presence of Failures}.
 
 
Cougar \cite{tong2009evaluation}, \cite{bonnet2001towards} is
another middleware applying database pattern in sensor network.
Cougar is extension of the Cornell Predator object-relational
database system. In Cougar system, there are two types of data:
stored data and sensor data. Signal processing functions in each
sensor node generate the sensor data, and date are communicated or
stored in local as relations in database system. Signal processing
functions are modeled by using Abstract Data Type in Cougar. In
Cougar, each type of sensor is modeled as a new Abstract Data
Type. Signal-processing functions are modeled as ADT functions
that return sensor data. Long-running queries are formulated in
SQL with little modifications to the language. To support the
evaluation of long-running queries, the query execution engine is
extended with a new mechanism for the execution of sensor ADT
functions.
 
Data service Middleware \textit{DsWare} \cite{hadim2006middleware}
is developed on a database approach but it can be also considered
as event based because of the way it is fitting in the sensor
networks. From the architectural point of view, it consists of
several modules: data storage, data caching, group management,
event detection, data subscription and scheduling. It uses SQL
language to manage the events. It has real-time execution
performance and it is considered as a very reliability middleware.
It can handle the dynamic nature of sensor network data. His
reliability relies on the fact that it can be serviced by a group
of geographically close sensor nodes. In the present form, it
doesn't support heterogeneity and mobility.
 
 
\textit{SensorWare} middleware \cite{boulis2003design} implements
distributed algorithms and at the same time can hide low-level
details from the application. It provides a way to share the
resources of a node among many applications and multiple users of
Wireless Ad-hoc Sensor Network. SensorWare is conducting the
dynamic deployment of the distributed algorithms into the network.
Usually this means that a distributed algorithm has to be
incorporated in several sensor nodes, which in turn means that
these sensor nodes have to be dynamically programmed. A
user-friendly and energy-efficient way of programming the nodes
keeps the user out-of-the-loop most of the time by allowing sensor
nodes to program their peers. From the architectural aspect, a
sensor node in SensorWare can be viewed in layers. The lower
layers are the raw hardware and the hardware abstraction layer
(i.e., the device drivers). An operating system (OS) is on top of
the lower layers. The OS provides all the standard functions and
services of a multi-threaded environment that are needed by the
layers above it. The SensorWare layer for instance, uses those
functions and services offered by the OS to provide the run-time
environment for the control scripts. The control scripts rely
completely on the SensorWare layer while populating around the
network. Static applications and services coexist with mobile
scripts. They can use some of the functionality of SensorWare as
well as standard functions and services of the OS. SensorWare is
evaluated through: Memory size measurements, Delay measurements
and Energy measurements and related tradeoffs.
 
The main objective of the middleware integration platform
\textit{Sensation} for WSN is to provide heterogeneity of between
different sensor network infrastructures attributed to the variety
of sensor, networking, or middleware technologies. Moreover, it is
designed in order to provide a developer-friendly programming
model that can further encourage the development of pervasive
applications. Heterogeneity is provided through the abstraction
layer. \textit{Senstation} has a high-level and intuitive
programming model for context-aware pervasive applications.
 
 
TinyDB \cite{molla2006survey} \cite{molla2006survey} is a query
processing middleware system based on TinyOS. TinyDB provides
power-efficient in-network query processing system for collecting
data from individual sensor nodes which reduces number of messages
that must be sent. This results in reduced energy consumption. It
has two different types of messages for query processing – Query
Messages and Query Result Messages. It also has Command Messages
for sending command to sensor nodes. While TinyDB provides nice
abstraction support and has good aggregation model, it does not
provide much functionality as part of middleware service. So most
of the services have to be provided in the applications running on
top of it.
 
 
\subsection{Event based middlewares}
 
\subsubsection{Publish/Subscribe oriented design middlewares}
 
Publish/Subscribed oriented middleware contains a set of subscribers and a set
of publishers. Subscribers can have access to publishers data streams through a
common database and they register their interest for events or pattern of events
and are subsequently asynchronously notified of events generated by publishers.
It is possible to have multiple subscribers for a single publisher, and multiple
publishers for a single subscriber \cite{eugster2003many},
\cite{azzara2013middleware}.
 

 
Middleware called \textit{PRISMA} is a resource-oriented publish/subscribe
middleware for WSN. The main goals of \textit{PRISMA} are to provide: (i)
programming abstraction through the use of REST interfaces, (ii) services,
encompassing asynchronous communication, resource discovery and topology
control, (iii) runtime support through the creation, configuration, and
execution of new applications in WSN, and (iv) QoS mechanisms to meet
applications constraints. In \cite{silva2014prisma}, the architecture of PRISMA,
its implementation in the Arduino platform, and a preliminary evaluation are
presented.
 
Epidemic Messaging Middleware for Ad hoc networks (\textit{EMMA}) is an
adaptation of Java Message Service (JMS) for mobile ad hoc environments which
architecture is also based on publish/subscribe design \cite{musolesi2005emma}.
 
 
\subsection{Service oriented Middlewares}

Service-orientation is a design paradigm to develop software or applications in
the form of services. Service-oriented computing (SOC) based on Service Oriented
Architecture (SOA) approaches which traditionally used in corporate IT systems,
are becoming applicable in the field of IoT due to its number of benefits
including abstractions to wrap the heterogeneous functionalities of things
within higher level services, service discovery and composition. In IoT
infrastructures, composed of ultra large numbers of networked,
resource-constrained devices, the discovery of services and adaptive composition
of services is a substantial challenge. The service-oriented middleware (SOM)
can tackle these challenges in IoT by supporting the service-oriented
interaction pattern through the provision of appropriate functionalities for
deploying, publishing/discovering and accessing services at runtime. SOM also
provides support for adaptive service compositions in case of unavailable
services.


A large number of works have been published on SOMs in IoT. These works can be
categorized as standalone service oriented middleware for
IoT~\cite{Hydra08,SOCRADES10 , UbiSOAP12, Servilla12, WuKong1-13, MoboIoT14} and
middleware services provided cloud computing's platform as a service
(PaaS)~\cite{ Carriots, Xively,SicstheSense}. In the following, we summarize
some key and recent works in each of these categories.

\textit{Hydra}~\cite{Hydra08, Hydra09a}, which is currently known as
LinkSmart~\cite{LinksmartHydra14} middleware, has been designed and developed to
allow developers to build ambient intelligence services and systems through a
SoA and model driven architecture approach. This middleware provides support for
using devices as services both by embedding services in devices and by proxy
services for devices. Its architecture contains a number of management
components including service manager, event manager, device manager, storage
manager, context manager, security manager, etc. � to that handles the various
tasks needed by the middleware. These are grouped into application elements and
device elements which are also consists of semantic layer, service layer,
network layer and security layer. Use of semantic web services to device levels
along with SoA in \textit{Hydra}, it can support syntactical as well as along
with semantic level interoperability. Along with number of functional
requirements (e.g. data management, event management, resource management) it
also supports dynamic reconfiguration and self-configuration, which are key
properties in any AmI application. With the support of resource manager, device
manager, and policy manager Hydra suits in resource constrained devices or
things. Distributed security and social trust components of the middleware
offers secure and trustworthy communication within devices. To fulfil security
and privacy goals it exploits virtualization and trust-worthy design and
implementation of web-service based mechanisms, enriched by semantic
resolution~\cite {HydraSecurity08}. Use of virtualization may introduce security
concerns (e.g. side channel attack) if it supports multi-tenancy. Also, use of
ontology based semantic security and interoperability solutions may not be
viable in IoT. Due to lack of standard ontologies, dealing with ontologies in
SoC is a challenging task and it will become worse in case of ultra-large scale
IoT.

 
 
\textit{SOCRADES} (Service-Oriented Cross-layer infRAstructure for Distributed
smart Embedded devices)~\cite{SOCRADES08, SOCRADES10} middleware abstracts
physical things as services using Devices Profile for Web Services (DPWS).
Instead of starting from the scratch it relies two earlier works~\cite{
SIRENA06,SODA06}. The main objective of \textit{SOCRADES} is to simplify the
management of underlying devices or things for the enterprise applications (e.g.
industrial automation).
Its architecture consists of an application services (e.g. event storage) and a
device services (e.g. device manager and monitor, service discovery, service
lifecycle management) part. Different components of these two parts aim to
fulfil different requirements of SOM. For instance, device services part's
service discovery component, a key contribution of \textit{SOCRADES} middleware,
discovers the services provided by the real-world devices or things while its
device management and monitor component does the resource management (e.g.
device access). It provides discovery on two levels, the device level and the
service level, which can employ either standard WS-discovery (for Web Services)
or a RESTful discovery mechanism (for RESTful services). Application services
part provides event management and storage. Also, \textit{SOCRADES} middleware's
Cross-layer Service Catalogue in between device and applications parts supports
service composition that may not be fully dynamic, as composition relies on
predefined building blocks. Role-based access control of devices communication
to middleware and back end services and vice versa works as a security solution
but limited to authentication only. Moreover, direct access to devices or their
offered services through this middleware raises the risk of privacy violations.

\textit{SenseWrap}~\cite{SenseWrap09} middleware combines the
Zeroconf~\cite{Zeroconf} protocols with hardware abstraction using virtual
sensors to offer a service-oriented and lightweight middleware for application
developers. A virtual sensor provides transparent discovery of resources (mainly
sensors) through the use of Zeroconf protocols. Applications can discover
sensor-hosted services through Zeroconf. \textit{SenseWrap} also provides a
standardized communication interface to hide the sensor-specific details from
the applications. This interface relies on sensor modeling and custom wrappers
(drivers) for each sensor model. Also, virtualization is limited to sensors not
actuators or computing. These issues makes it unsuitable in IoT environments
considering scale and heterogeneity in infrastructure and applications.

\textit{MUSIC}~\cite{MUSIC09} middleware provides a self-adaptive
component-based architectures to ease the building of systems in ubiquitous and
SoA environments where numerous and unexpected changes may happen in context in
different perspectives (e.g. service providers, service consumers). In
particular, middleware work in this work focuses on changes in the service
provider site in order to hook up interchangeably components and services
providing the functionalities defined by the component framework. To support
QoS-aware and context-based dynamic adaptation, \textit{MUSIC} middleware
architecture comprises of components like context manager, QoS manager,
adaptation manager, plan repository, SLA negotiator and monitoring, service
discovery and these components provides different functionalities for the
middleware. For instance, in planning-based adaptation, planning (available in
plan repository) is typically triggered by context changes detected by context
manager. With the support of these components, the dynamic adaptations work
automatically to optimize the application utility in a given context. Context
may contain a lot of private and sensitive data (e.g. location or interests of a
user) and thereby increases the risk of privacy leakage.

\textit{TinySOA}~\cite{TinySOA09} is a service-oriented architecture that offers
a high-level abstraction for the development of WSN applications. It provides a
simple service oriented API through which application developers can access WSN
resources from their applications. Along with hiding the low-level technical
details of the WSN hardware and communication for the application developers, it
easily integrate Internet applications with WSN and allow them to collect
information from the sensors. TinySOA employs simple and deterministic
mechanisms for WSN resource (e.g. sensors node) registration and discovery. It
uses a gateway component to bridge between a WSN and external applications and
web services to allow applications to access WSN. Other than few basic
functional requirements (e.g. abstraction, resource discovery and management) it
does not consider non-functional requirements.

\textit{SensorsMW}~\cite{SensorMW10} is an adaptable and flexible SOM for QoS
configuration and management of WSN. It abstract WSNs as a collection of
services for seamless integration into enterprise information systems, where
applications can gather sensed information using web services and easily
configure to efficiently collect information from WSN. WSN resources are managed
for guaranteeing certain QoS according SLAs. Importantly, it offers applications
to reconfigure and maintain the network during its lifetime. In this context,
high-level applications are require to control and trade-off between the
conflicting issues (e.g. lifetime and sampling rate). To achieve this, WSN
resources (nodes) need to be accessed to change their configurations to fit with
the applications requirements which \textit{SensorsMW} provides an abstract way
for the high-level applications. Resource reconfiguration and management needs
resource discovery, especially in mobile IoT where resources are dynamic, is not
addressed here. Also, in critical applications this reconfiguration may fail as
their strict QoS requirements may not allow any trade-off amongst necessary
resources.


\textit{SENSEI}~\cite{SENSEI10} middleware aims to develop an architecture for
the future and real world Internet including IoT. It is one of the earliest work
that includes the context model, context services, actuation tasks and dynamic
service composition of both primitive and advanced services for real world
Internet. Main component of this middleware is the resource layer that is
sandwiched between application layer and communication services layer. The
resource layer contains all necessary functions that facilitate the interaction
of applications and services with the resources. The resource component of the
resource layer provides a unifying abstraction for simple devices (e.g. sensors,
actuators, processors or software components). Support services component
enables discovery, composition and dynamic composition of resources (e.g.
services) and support long term interactions. The dynamic composition and
instantiation of new services are facilitated by the semantically-rich models
and descriptions of sensors, actuators and processing elements. The resource
layer also contains functions for privacy and security (e.g. authentication).
Similar to \textit{Hydra}~\cite{Hydra08, Hydra09a}, due to lack of standard
ontologies (e.g. context modeling, semantic modeling, and domain knowledge),
\textit{SENSEI} will struggle in IoT because of its ultra-large scale
infrastructure and diverse applications.



The~\textit{ubiSOAP}~\cite{UbiSOAP12} is a service-oriented middleware to
provide seamless networking of web services that may be deployed on various
devices including mobile devices. Its architecture consists of a multi-radio
networking layer and a web services (WS) oriented communication layer. The
multi-radio networking layer manages heterogeneous network resources using a
network-agnostic addressing scheme and offers network-agnostic connectivity to
services. This layer also offers the functionality for the QoS-aware (e.g.
energy consumption, availability) network selection. The WS-oriented
communication layer offers point-to-point (P2P) communication that supports
legacy WS and thus transparently bring the added value of today's ubiquitous
networking environments to existing services. Ubiquitous service discovery
(ubiSD-S) layer on top of UbiSOAP provides dynamic, interoperable, context-aware
service discovery. In general, UbiSOAP is a lightweight SOM that offers resource
management (e.g. device and network management) and network level
interoperability by supporting heterogeneous networking device and technologies.
Lack of context-awareness in~\textit{ubiSOAP} can be an issue, as it is key in
adaptive and autonomous behavior of the things. It may also suffer due to lack
of security and privacy solutions at the middleware level.


\textit{Servilla}~\cite{Servilla12} is a middleware that facilitates application
development in heterogeneous WSNs by exploiting SOC to decouple
platform-specific code from platform-independent applications. It structures
applications as platform-independent tasks that are dynamically bound to
platform-specific services. \textit{Servilla's} architecture consists of a
virtual machine (VM) and a service provisioning framework (SPF) and runs on
individual sensor nodes in a WSN. The VM executes application tasks while the
SPF consists of a consumer (SPF-consumer) that discovers and accesses services,
and provider (SPF-provider) that advertises and executes services. A VM is used
abstract heterogeneity in instruction sets of WSN devices and to compile tasks
using the VM's uniform and platform-independent instruction set. The VM also
offers the dynamic deployment and mobility of tasks with the support of dynamic
service binding and the novel binding semantics. \textit{Servilla's} components
including code manager, service management (e.g. service binding, service
finder) along with native implementation offer energy and memory efficiencies. A
specialized service description language facilitates flexible matching between
applications and services residing same or different devices, but this
specialized language requirements could limit the wide spread adoption of this
middleware in IoT. Moreover, individual sensor level access in this middleware
could introduce privacy violations and security threats.


\textit{KASOM}~\cite{ KASOM12} is a Knowledge-Aware and Service-Oriented
Middleware (KASOM) for pervasive embedded networks, especially for WSANs
(Wireless Sensor and Actuator Networks). Its architecture consists of three
major subsystems: framework services (e.g. security, runtime manager),
communication services (e.g. resource monitor), and knowledge management
services (e.g. service composition rules, context resources). These services
allow addressing a SOA for pervasive environments using registration, discovery,
composition, and orchestration of services. Most of these services are
established on complex reasoning mechanisms and protocols based on the WSAN's
contextual model, which represents a semantic description of low and high level
resources of the WSAN. All these services provide a number of tools for the
developers to facilitate the design and deployment of services in pervasive
network. Real life implementation of it in hospital and health management shows
the potential of KASOM in terms of response time, efficiency and reliability.
However, due to in-network agents provided predefine service composition rules,
it will fail to provide dynamic service composition in mobile and resource
constrained IoT infrastructures. Moreover, proposed security solution in KASOM
by access control is limited to authentication only.
\textit{CHOReOS}~\cite{CHOReOS12, CHOReOS13} is SOM that enables large scale
choreographies or compositions of adaptable, QoS-aware and heterogeneous
services in the Future Internet like environment including IoT. It addresses the
scalability, interoperability, mobility, and adaptability issues of the Future
Internet through novel approaches like scalable probabilistic thing-based
service registries and discoveries~\cite{MoboIoT11,MoboIoT14}. The CHOReOS is
composed of four components: eXecutable Service Composition (XSC) to coordinate
the composition of services and things, eXtensible Service Access (XSA) to
offers mechanisms to access services and things, eXtensible Service Discovery
(XSD) to manage protocols and processes for discovery of services and things,
and the Cloud and Grid middleware to manage computational resources and drives
the deployment of choreographies.\textit{MobIoT}, a key component of CHOReOS
middleware, is presented separately in~\cite{MoboIoT11,MoboIoT14} as a
thing-based SOM for the Mobile IoT. Unlike most existing SOMs~\cite{ KASOM12,
UbiSOAP12}, its thing-based probabilistic service discovery and registration and
look-up protocols and algorithms scales well in dynamic mobile IoT. Moreover,
semantic thing-based service composition and the algorithms allow
\textit{MobIoT} and \textit{CHOReOS} to execute automatically, with no
involvement from developers or end-users which is highly desirable in IoT,
especially in M2M communications. Considering the heterogeneity and lack of
standard in ontologies, ontology based semantic support for application
development and service composition in diverse applications and heterogeneous
infrastructures of IoT is a very challenging task.

\textit{MOSDEN} �Mobile Sensor Data Processing Engine~\cite{MOSDEN14} is a
middleware to support sensing as a service model~\cite{SensingService14}. It is
built on top of GSN- an existing cloud assisted middleware~\cite{GSN06}. Using
plugin architecture it improves the scalability and user friendliness (e.g. easy
to use) of the solution, considering that plugins for heterogeneous devices can
be built by anyone and available in easily accessible places (e.g. Google play).
\textit{MOSDEN} added a plugin manager and a plugin layer in GSN to support and
manipulate plugins. It also eliminated sensor dependent individual wrapper from
GSN by a single generic wrapper to handle the communication. As it relies on
GSN, like GSN it will suffer in IoT in terms of resource/service discovery and
service composition. GSN employs traditional discovery
approaches~\cite{ServiceDiscovery08} and predefined composition rules available
in virtual sensors that may not work in IoT's dynamic and ultra large networks.



Along with the aforementioned standalone SOM proposals, a large number of cloud
based IoT platforms are commercially available~\cite{CloudIoT1}. To provide an
impression of the field in the following we summarize few of these and for
others, readers are referred to~\cite{CloudIoT1} and references therein.

\textit{Xively}~\cite{Xively} is a PaaS that provides middleware services to
create exciting products and solutions on the IoT. Public cloud based
\textit{Xively} offers developers standards-based directory services, data
services and business services.
Directory services help to find appropriate object with appropriate permission.
Data management services along with high performance and time-series database
allows to store and retrieve easily and reliably. Its web-based tools simplify
the data or control with other applications complexities of IoT development.
Business services includes device lifecycle management service including device
provisioning.  Device lifecycle management along with real-time message bus in
\textit{Xively} provides large scale and real-time deployments in IoT.
Importantly, \textit{Xively} provides end to end security over the entire
platform to ensure IoT solution integrity. \textit{Xively} services are free of
charge for developers but it offers a limited amount of API calls per minute.
Lack of storage security or data at rest and trust~\cite{IoTCloudConcerns15} can
be an issue in IoT applications. It supports multiple data formats for input and
output, like XML, JSON, CSV and EEML. However, it doesn't homogenize the
incoming data. Also, it supports a list of software and hardware combinations
needed to develop IoT applications but its interoperability is limited within
this list.
 
 
\textit{CarrIoT}~\cite{ Carriots} is a cloud service oriented middleware for
IoT, especially for M2M communications with three main focus: (i) simple, fast
and cost effective M2M application development, (ii) scalability and (iii) easy
to use. It has rich set of APIs and powerful SDK. Great advantage of
\textit{CarrIoT} is that it supports immediate scalability, which enables the
project growing and multiply the number of devices. Users can, for example, put
triggers on various stages of the data processing cycle, from receiving the data
to persisting it into the database. Triggers and other actions can be easily
configured in the provided Web GUI\cite{puhringercloud}.
However, \textit{CarrIoT} are not free. There is a free starting plan which lets
you manage up to 10 devices but if you need more than that you will have to pay.
Another aspect to take into account is that the users can't share their data
publicly. Also, just like Xively's platform, they don't standardize the incoming
data and does not guarantee storage security, and offers limited
interoperability~\cite{IoTCloudConcerns15}.


 
\textit{Echelon}~\cite{ Echelon} is an IIoT platform with a full suite of chips,
stacks, modules, interfaces and management software for developing devices,
peer-to-peer communities and applications delivered via the IzoT Device Stack,
IzoT server stack and FT 6000 EVK. Unlike consumer IoT platforms it addresses
the core requirements for the IIoT, including autonomous control,
industrial-strength reliability, support for legacy evolution and exceptional
security. Similar to Xively, \textit{CarrIoT} and other many cloud platforms,
its interoperability is limited within Echelon's and list of other hardware.
Being private cloud, its security is better than Xively, but trust still is an
issue for sensitive IIoT applications.

Aforementioned works are not the exhaustive list of available SOMs in IoT.
Considering the space limitation, we have covered a number of recent (from 2009)
representative works to present the stat-of-the-art in this particular category
of IoT middleware. Also, we did not explicitly few works which have been
extended by the presented works. For instance, we have not included middlewares
in~\cite{SIRENA06,SODA06} as they have been extended in~\textit{SOCRADES}. A
number of service oriented middlewares, especially designed for WSNs are available in
~\cite{ Carisma03, SI2-06, OASiS06, MidCASE07, SensorWeb2.0, Sstreamware08,
SOAMenuF08, USEME08, DySSCo08, SATWare09, MiSense09,B-VIS09, WhereX10, SOMWSN10,
SOMM11}. Most of these works are summarized in~\cite{SOMWSNs11}. For the detail,
readers are referred to~\cite{SOMWSNs11} and references therein. These work
support one or more of the IoT middleware requirements but not all identified
earlier.


 
As SOC by nature support abstraction and do not explicitly deals with code,
existing SOMs do not explicitly consider abstraction and code management. The
majority of the existing service oriented middlewares are WSNs centric and their
scale is limited to WSNs scale, which is typically in the range of thousands,
much less than the ultra-large scale (billions) of IoT. Importantly, most of
these middlewares' resource discovery and management, predefined and
deterministic composition mechanisms will not scale well in ultra-large and
dynamic IoT environments. Lack of global and standard ontologies, semantic
interoperability of the existing SOMs will not suit well in IoT. Most existing
standalone SOMs, which are considering security, are offering limited scale
security by supporting authentication not integration and availability. On the
other hand, cloud platform supported middlewares struggles in storage security
and trust.


 
 
%Other examples are Mires, Corba...
 
 
 
 
\subsection{Virtual machine oriented design}
 
Virtual machine (VM) oriented design provides virtualisation of real hardware,
intermediate program representation or bytecode interpretation
\cite{azzara2013middleware}. Virtual machine oriented middleware placed between
the operating system and applications is one type of VM design SOFTWARE. Another
group is system level VMs that fully replace the operating system, often
providing advanced functionality such as concurrent programming support. By
using virtual machines the same application runs transparently in any platform
supporting the interpreter. One of the problems related to VM oriented design is
related to the lack of resources in WSN nodes. Another disadvantage of this
approach is that virtual machine becomes oriented to special types of
application, thus reducing flexibility and the possibility of re-tasking.
 
The agent-based middleware for social networking services (SNS)
(\textit{MAgNet}) \cite{basuga2009magnet} is a proof-of-concept
prototype that enables SNSs for users in mobile network domain.
Services provided by the MAgNet middleware can be roughly divided
into two groups: creating and managing groups of users and
planning group events. Current set of implemented services is used
as the MAgNet proof-of-concept prototype but it possible to extend
the middleware for provisioning more SNSs.  A social network is a
structure comprising of nodes connected to each other by using
various criteria. It is described as a graph $G = (V, E)$. The
nodes (i.e., vertices $V$) represent people or organisations
connected through various criteria while the edges in the graph
($E$) represent those connections. A software agent is a computer
program that autonomously acts on behalf of its user. The most
important property of an agent is its autonomy, meaning that the
agent can act without direct intervention from its user and has
control over its own actions and internal state. The architecture
of the proposed MAgNet middleware is designed as a MAS system
which contains three types of software agents: Graphical User
Interface Agent (GA), User Agent (UA) and Social Agent (SA). The
UA and its accompanying GA are assigned to a single user. The UA
is designed to represent the user within the MAgNet middleware.
The GA provides an interface between the user and his/her UA: it
basically enables the user to access all of middleware
functionalities and communicate with other users by presenting all
relevant information and personal messages to the user via
graphical interface. And the SA provides the support and
coordination needed for providing SNSs within the developed MAgNet
middleware. The \textit{MAgNet} middleware was developed using
Java Agent DEvelopment framework (JADE)5, a framework for
developing agents and MASs in compliance with Foundation for
Intelligent Physical Agents (FIPA)6 standards. All agents within
the middleware communicate by exchanging Agent Communication
Language (ACL) messages. Each ACL message, besides carrying
information for the receiving agent, has a type that defines its
function (e.g., inform, request, cancel, reject or accept). In
addition to message types, each message can be assigned with a
conversation identifier (CID). The CID facilitates the detection
and filtering of incoming ACL messages. Filtering ACL messages
based on their CID and message type simplifies the communication
between agents. Each agent has a number of tasks. Every task is
implemented as agent?s behaviour, which is basically a Java class.
Adding behaviour to the agent tells an agent to execute the task
which the behaviour represents. Since multi-threads are supported
in Java, an agent can execute several behaviours concurrently.
This allows having a single Java thread per agent which is
especially useful in environments with limited resources such as
mobile phones.
 
 
\textit{Mate} \cite{hadim2006middleware} is a middleware for WSN
which uses a virtual machine approach as an abstraction layer to
implement its operation and tackle the challenges. The general
idea of \textit{Mate} is to provide new programming paradigms to
overcome constraints such as limited bandwidth and network
activities' large energy draw. \textit{Mate} proposes a spectrum
of reprogrammability, from adjusting simple parameters to
uploading complete program updates using a virtual machine.
Sending a single bit of data can consume the same energy as
executing thousands of instructions to produce that bit of data.
\textit{Mate} is a byte code interpreter that runs on TinyOS. Mate
uses codes broken into capsules of 24 byte-long instructions. This
benefits large programs, which are made up of multiple capsules
and, thus, easily injected into the network. Mate's key components
are the VM, the network, the logger, the hardware, and the
boot/scheduler. Using a synchronous model that begins execution in
response to an event such as a packet transmission or a time out,
Mate avoids message buffering and large storage. The synchronous
model makes application- level programming simpler and far less
prone to bugs than dealing with asynchronous event notifications.
Another Mate functionality is infection or network updates
achieved by adding a version number to the capsule. So, comparison
takes place at the neighbors, followed by installation of the new
version. This process cascades with hop-to-hop communication.
 
\textit{Muse} is VM based middleware for using sensors
effectively. MUSE is a middleware-architecture to support the
development of contextual real-time applications and services for
sensor-equipped smart environments. MUSE, like the
TUIpist-Framework, uses Jini infrastructure technology as service
platform. The sensor service in MUSE is organized similarly to the
sensor components of TUIpist; the fusion service provides similar
functionality like the ApplicationMediator in the
TUIpist-framework. \textit{Muse} has the problem with scalability
and does not offer an opportunity for persistent data storage e.g.
for later evaluation.
 
\textit{SwissQM} (Scalable WIreleS Sensor network Query Machine)
\cite{mueller2007swissqm} sis virtual machine middleware for WSN
built to address lack of data independence and poor integration
with the higher layers of data processing chain. \textit{SwissQM}
combines a virtual machine at the sensors with a powerful gateway
as access point to the system. The flexibility of \textit{SwissQM}
opens up the doors to sensor networks with richer functionality,
data model independence, optimised performance, and smooth
integration into the rest of the architecture. SwissQM supports
adaptability (e.g., push down strategies), multiple users and
applications, high query turn-around, extensibility (e.g., user
defined functionality), and optimised use of resources. SwissQM is
based on a specialised virtual machine that runs optimised byte
code rather than queries. As a result, SwissQM does not make any
assumptions about the query language used (e.g., SQL or XQuery),
about the deployment strategy of the underlying sensor network
(e.g., one single network or multiple networks), and can easily
provide highly efficient multi-user support. SwissQM has been
designed with several requirements in mind: separation of sensors
and external interface; dynamic, multi-user, multi-programming
environment; optimised use of the sensors; and extensibility.
SwissQM should not implement any particular query language. The
programming model should be independent of the query lan- guage
used. It must also be dynamically adaptable. As a result, the
sensor nodes should not contain application specific functionality
(e.g., the ability to parse SQL or join operators). Such
functionality is treated as a dynamically deployable extension.
SwissQM should not impose restrictions on the query submission and
change rate, nor in the number of queries that can be run
concurrently (beyond the inherent limitations of the underlying
hardware). The only processing at the sensor nodes should be that
related to capturing, aggregating, and forwarding data. Anything
else should be there only because it has been pushed down from
above. This increases the memory available for data and leaves
room for more queries and/or more sophisticated processing such as
event generation or user defined functions. SwissQM should be
programmable to include the ability to implement user defined
functions and the ability to push down functionality from higher
data process- ing layers. Extensibility also refers to SwissQM
itself: It should be possible to dynamically extend and modify the
behaviour of SwissQM as needed.
 
%The \textit{Squawk} Java Virtual Machine is the result of an effort to write a J2ME CLDC compliant JVM in Java that provides OS level mechanisms for small devices, easing porting %and debugging of the VM. The obser- vation was that most JVMs are written in the C and C++ languages, even though complex processes performed by the JVM can be bet-
 
 
 
% \subsection{Hybrid oriented design}
 
\subsection{Tuple based}
 
LIME (Linda in a mobile environment) \cite{murphy2001lime} is the middleware for
mobile networks developed like a response to the software engineering challenge
posed by the advent of mobility. LIME borrows and adapts the coordination model
made popular by Linda \cite{gelernter1985generative}. In Linda, processes
communicate through a shared tuple space that acts as a repository of elementary
data structures, or tuples. It is agent-based and application-driven middleware.
Throught its \textit{Strong and Weak Reactions}, LIME can handle high-priority
or low-priority jobs. One of the most important characteristics of LIME is the
class \textit{LimeTupleSpace} which stands for the shared tuple space.
In the implementation of LIME, presented in \cite{murphy2006lime}, agents are
single-threaded and only the thread of the agent that creates the tuple space is
allowed to perform operations on the \textit{LimeTupleSpace} object; accesses by
other threads fail by returning an exception. Initially, all tuple spaces are
created as private, and are becoming shared by enabling the instance
\textit{setShared}.
LIME can achieve only partial context-awareness, since it doesn't have awareness
of the system configuration.
 
 
\textit{UbiROAD} is semantic middleware for context-aware smart road
environments \cite{terziyan2010ubiroad}.
Its purpose is to deal with the interoperability between the in-car and roadside
devices produced and programmed by different providers. Interoperability is
ensured by two layers: data-level interoperability and functional protocol-level
interoperability and coordination. With respect to interpretability challenge,
UbiRoad needs to deal with three problems: technical interoperability, semantic
interoperability and pragmatic interoperability. With regards to the technical
interoperability, it means that it should be capable to interoperate devices,
protocols and other technical standards.
Semantic interoperability means being capable of various system components to
treat and interpret exchanged data and information identically and share a
common understanding of it. Pragmatic interoperability means being capable make
system components to capture willingness of partners to collaborate or, more
generally, to capture their intent. UbiROAD is on top of the future smart
traffic environments but also between the smart road device layer and the future
service oriented architectures. Heterogeneity with respect to the nature of
components, standards, data formats, protocols, etc. can create significant
obstacles for interoperability among the components of ubiquitous systems.
UbiRoad is designed to easily deal with this issue. Another major issue of
UbiROAD is to guarantee high level of safety. It is self-adaptive by deploying
distributed agents. General idea of this middleware is to ensure
context-awareness, adoptable and reconfigurable composition.
It is achieved by customization, personalization, added value, dynamicity and
autonomy of services. Autonomous management is achieved by autonomous trust
management via semantic annotation. Semantic concepts are interpreted using
trust ontologies.
 
\textit{TeenyLIME} \cite{costa2006teenylime} is part of a broader
vision concerning the application of tuple spaces to mobile and
wireless sensor networks. It is a middleware for sensor networks
based on the tuple space model made popular by Linda.
\textit{TeenyLIME} is mostly based on the LIME middleware
developed for mobile ad hoc networks (MANETs). Like LIME,
\textit{TeenyLIME} operates by distributing the tuple space among
the devices, transiently sharing the tuple spaces contents as
connectivity allows, and introducing reactive operations that fire
when data matching a template appears in the tuple space. One
significant design decision in \textit{TeenyLIME} is to restrict
transient sharing only to the tuple spaces of one-hop neighbors.
This choice is not only energy-conscious, but is also in line with
similar research efforts in the field. \textit{TeenyLIME} can
enable a number of application-level uses thanks to the control of
the one-hop neighborhood around device, augmented with the
powerful and expressive primitives. A version of
\textit{TeenyLIME} for TinyOS is designed and implemented. The
application programming interface (API) id designed for the nesC
language. In addition to providing the usual Linda primitives and
those added by LIME, \textit{TeenyLIME}  API provides features
introduced explicitly to address the specific requirements of
WSNs. From the architectural point of view, \textit{TeenyLIME}
consists of three TinyOS components wired together in the
configuration. First component is the \textit{TupleSpaceM} that is
invented to delegate the actual operation to one of the two other
components: the \textit{LocalTeenyLIME} or the
\textit{DistributedTeenyLIME} components. The
\textit{LocalTeenyLIME} component stores the tuples output to the
local tuple space, and performs the actual matching process for
query operations. The \textit{DistributedTeenyLIME} component is
in charge of implementing the operations whose scope is different
from the local device.
 
\textit{TinyLime} \cite{curino2005mobile} is middleware for WSN,
which is build no the background of \textit{LIME} and
\textit{LINDA} middlewares. Apart from the transiently shared
tuple space used as a storage for sensed data (that is introduces
already in \textit{LINDA}), \textit{TinyLime} also posses a
component n addition to agents and hosts the motes. Motes are
scattered around and communicate with base stations only when the
latter move within range. A mote is not visible through TinyLime
unless it is connected to some base station. When this is the
case, the mote is represented in the model much like any other
agent residing on the base station host (and therefore ?connected?
to it), with its ITS containing the set of data provided by its
sensors. Data aggregation within \textit{TinyLime} can be gathered
in two different levels: globally, over the values sensed by
multiple sensors, or locally, over the values sensed by a single
sensor. A sensor can be preprogrammed to take samples at regular
intervals throughout its lifetime. Alternately, it can start in
passive mode, and it is up to the application programmer to
explicitly activate sampling for a given number of epochs, after
which the sensor automatically switches itself back into passive
mode, without additional communication. On standard hosts,
\textit{TinyLime} is implemented as a layer on top of
\textit{Lime} without requiring any modification to it,
reasserting the versatility of the Lime model and middleware. Main
architectural components of \textit{TinyLime} are
\textit{MoteLimeTupleSpace}, \textit{MoteLimeTuple} and
\textit{MoteLimeTemplate}.
 
 
\textit{TS-Mid} \cite{de2008ts} is another tuple space based
middleware for WSN and it threats with disconnections naturally by
providing an asynchronous and decoupled communication style on
both time and space, i.e., providers and receptors do not need to
be available in the same instant. \textit{TS-Mid} adopts the
JavaSpaces specification, including operations to write, read and
take entries, together a mechanism for events notification in the
tuple space and the creation of replicated JavaSpaces service in
the WSN. Each logical region is viewed as a distributed tuple
space, where parts of the tuple space are physically located in
each sensor node. In practical terms, after defining which sensors
nodes belong to each logical region, the sink node may transmit
requisitions to client application for logical region leader node.
The leader node is responsible for directing the requests of
execution of operations on the tuple space to other members of
region, collecting members reply, computing and transmitting
results back to the sink node. Data transmission is always
accomplished through tuples on the one-hop routing algorithms.
\textit{TS-Mid} has been implemented atop of TinyOS. From the
architectural point of view, \textit{TS-Mid} consists of three
main nesC modules, namely LRTSpaceM, AggregateM and TSpaceM. These
components are wired together in the configuration LRTSpace.
Module called LRTSpaceM implements the operations of interface
TSpace related to creation of logical regions and additionally, it
stores information about these regions in a list. Module TSpaceM
is responsible for implementing the operations defined in
interface TSpace related to the access to local tuple space of
each sensor node belonging to logical region. This component
stores entries (tuples) in local tuple space and executes the
matching process for reading operations. Module AggregateM
provides the interface Aggregate. This interface defines
operations to use the data aggregate service of the logical region
according to the criteria established in the reading operations.
The communication between Java applications and the sink node
(through a serial port) is provided by the GenericComm module of
TinyOS. Modules LRTSpaceM and TSpaceM use the module PacketHandler
that provides Send and Receiver interfaces to send and receive
messages, respectively.
 
 
\subsection{Application driven}
 
% Middlewares:  AutoSec, MiLAN, Adaptive Middleware, MidFusion
The aim of application driven approach is to provide a better resource management (i.e., QoS support)
compared to other variants by implementing an architecture which reaches the network protocol stack.
This offers the possibility to fine-tune the network based on the application requirements. However, this
approach lacks in support for platform (e.g., operating systems and hardware) heterogeneity
because of the tight coupling between applications and middleware layer that creates only
specialized middleware solutions\cite{hadim2006middleware}.
 
% AutoSec
An example of application driven middleware is AutoSec (Automatic Service Composition)\cite{han2001autosec},
which uses dynamic service brokering for resource management in a distributed system. This is done by appropriate
combination of information collection and resource provisioning policies based on current system conditions and
application requirements.
 
% Adaptive Middleware
Another example is Adaptive Middleware\cite{huebscher2004AdapMid} which explores the commitment between
resource spending and quality during information collecting. The main goal is to decrease the transmissions among
sensor nodes without compromising the overall result.
 
% MILAN
MiLAN (Middleware Linking Application and Networks)\cite{milan2004middleware} is also similar to Adaptive Middleware scheme.
However, MiLAN allows applications to specify their QoS requirements and adjust the network characteristics to increase
application's runtime while still meeting those requirements. The adjustments are made based on information collected
from the application, the user, the network and the overall system. The application provides the QoS requirements
over time and how to satisfy these using a combination of sensors. The overall system and the user then decide upon the
importance of different applications, while the network provides information the available resources.
Both Adaptive Middleware and MiLAN require knowledge about the exact sensors.
In dynamic and pervasive computing environments, the number and types of sensors available to the
applications may vary. It is impractical to include knowledge about all the different sensor nodes that
an application can potentially make use of. Moreover, all these sensors come at various levels of cost
and benefit to the application.
 
% MidFusion
MidFusion\cite{alex2008midfusion} builds on the concepts presented in MiLAN and Adaptive Middleware and
aims to avoid having knowledge about the exact sensors available by using principles of Bayesian and Decision
theory to provide a portable abstraction to application. In addition to MiLAN and Adaptive Middleware,
MidFusion is using the information aquisition cost to select the best set of sensors or sensor agents.
 
\begin{table*}[h!]
 \caption{Summary of the IoT middlewares in terms of supported functional
 requirements}
\label{tab1}
 \centering
{\scriptsize \begin{tabular}{ |l|l|l|l|l|l|} \cline{1-6}
\multicolumn{6}{|c|}{Functional requirements}\\
\hline
&Resourse Discovery & Resource Management & Data Management & Event Management & Code Management \\
\hline
%\multicolumn{6}{|c|}{Event-based (Publish/Subscribe)}\\
\multicolumn{6}{|c|}{Event-based}\\
\hline
Prisma  & DD& RM& DPDA& SN & CA\\
\hline
Emma &DD&RM & DA & NS& CA\\
\hline
Hermes &DD & NI & DPDF & LN& NS\\
\hline
Green &DD, LB& RM& DS, DPDF & LN& CM\\
\hline
%\multicolumn{6}{|c|}{Event-based (Message-oriented)}\\
%\hline
Mires & DD & NS& DPDA& LN& NS\\
\hline
SensorBus & DD &  NS& DPDA& LN& CA\\
\hline
%MQSeries & N/A &  N/A& Fully& Partially& Partially\\
%\hline

%%%% Raz
\multicolumn{6}{|c|}{Service-oriented}\\
\hline
Hydra & DD,DeD, SD & RA, RM, RCP & DS & SS & NS\\
\hline
SOCRADES~\cite{SOCRADES10} & DD,DeD, SD  & RA, RM, RCP & NI & LS & NS \\
\hline
SenseWrap~\cite{SenseWrap09} & DD, DeD, SD  & NI & NI & SS & NS \\
\hline
SENSEI~\cite{SENSEI10} & DD, DeD  & RM, RA, RCA & DS,DPA & NI & NS \\
UbiSOAP & Yes & Yes& Yes& Partially& Partially \\
\hline
Xively  & Yes & Yes & Partially & Partially & Partially\\
\hline
%&Resourse Discovery & Resource Management & Data Management & Event Management & Code Management \\
CarrIoT & Yes & Yes & Partially & Fully & Fully\\
\hline
SicsthSense & N/A & N/A & Yes &Partially & Partially \\


\hline
%% 

\multicolumn{6}{|c|}{Component based}\\
\hline
\multicolumn{6}{|c|}{Agent-based}\\
\hline
ActorNet & Yes & Yes & Fully & N/A& Fully\\
\hline
Agilla & Yes & Yes & Fully & N/A& Fully\\
\hline
AFME & Yes & Yes & Fully & N/A& Doesn't support\\
\hline
Impala & Yes & Yes & Fully & N/A& Fully\\
\hline
Maps & Yes & Yes & Fully & N/A& Doesn't support\\
\hline
Maspot & Yes & Yes & Fully & N/A& Fully\\
\hline
Smart Messages Project & Yes & Yes & Fully & N/A& Fully\\
\hline
TinyMAPS & Yes & Yes & Fully & N/A& Fully\\
\hline
\multicolumn{6}{|c|}{Tuple space approach}\\
%&Resourse Discovery & Resource Management & Data Management & Event Management & Code Management \\
\hline
LIME & DD &NS&  DS& NI& CM\\
\hline
UbiROAD & CD& RM& DS & NS& NS\\
\hline
TeenyLIME & DD & RM&  DPDA, DS& LN&CM \\
\hline
TinyLime & CD& NS& DPDA& NS&CM\\
\hline
TS-Mid & DD & RM & DPDA& NS&NS\\
\hline
\multicolumn{6}{|c|}{Virtual machine approach}\\
 \hline
%&Resourse Discovery & Resource Management & Data Management & Event Management & Code Management \\
MAgNet & Yes& Yes &Doesn't support & No& Partially\\
 \hline
Mate &Yes &  Yes& Doesn't support & Yes& Partially\\
 \hline
Muse &Yes &  Yes& Doesn't support & No& Partially\\
 \hline
SwissQM  &Yes &  Yes& Doesn't support & No& Partially\\
 \hline
\multicolumn{6}{|c|}{Database approach}\\
\hline
SINA & NS & RM & DPDA & NS & NS\\
\hline
IrisNet & DD,LB & RA & DPDA, DPDF  & NS & CM \\
\hline
HyCache & DD & RM & DS  & NS & NS \\
\hline
GSN & DD & RA & DS, DPDF & LN & CA\\
\hline
KSpot\textsuperscript{+} & LB, DD & RM & DPDA & NS & NS\\
\hline
Cougar & DD & RM & DS & NS & NS \\
\hline
DsWare & NI & NI& DS & LN& DA\\
%\hline
%SensorWare & No & No & Fully& Yes & Fully\\
\hline
Sensation & NI & NI & DS& NS& CM\\
\hline
TinyDB & DD& NI& NS& DPDA&  NI \\
\hline
\multicolumn{6}{|c|}{Application driven approach}\\
\hline
Milan & Partially & Yes & Partially & N/A & N/A\\
\hline
MidFusion & Partially & Yes & Partially & N/A & N/A\\
\hline
Adaptive Middleware & Partially & Yes & Partially & N/A & N/A\\
\hline
AutoSec & Partially & Yes & Partially & N/A & N/A\\
\hline
%\hline
%\rowfont{\tiny}
 \multirow{2}{*}{\textbf{Legend}}  & Centralised Discovery (CD) & Resource Allocation
 (RA) & Data storage (DS)& Supported  & Code Migration (CM) \\
%\hline
%\rowfont{\tiny}
 & Distributed Discovery (DD) & Resource Monitor (RM) & Data Preprocessing
 (DP)
 &\hspace{1mm}-Large Scale (LS) & Code Allocation (CA) \\
%\hline
 & Device Discovery (DeD) & Resource Composition (RC) & \hspace{1mm}-Aggregation
 (A) &\hspace{1mm}-Small Scale (SS)& \\

  No Information (NI)& Network Discovery (ND)    & \hspace{1mm}-Adaptive (A)& 
  \hspace{1mm}-Compression(C) & &  \\

Not supported (NS)&Service Discovery (SD) & \hspace{1mm}-Predefined (P) & \hspace{1mm}-Filtering (F) &                        &\\
& & Resource Conflict (RCL) &  &                       
&\\

\hline
\end{tabular} }
\end{table*}

 
 
%%%%%%%%%%%%%%%%%%%%%%%%%%%%%%%%%%%%%%%%%%%%%%
%%%%%%%%%%%%%%%%%%%%%%%%%%%%%%%%%%%%%%%%%%%%%%
 
 
%%%%%%%%%%%%%%%%%%%%%%%%%%%%%%%%%%%%%%%%%%%%%%
\subsection{Modular based}
 
 
% Citations to add
%
% Citation for AFME
%
% Middlewares: Impala, Smart Messages, Agilla, AFME, ActorNet, MAPS, MASPOT, TinyMAPS
In this approach, applications are divided as modular programs to facilitate injection and distribution through
the network using mobile agents for collecting local data. In their seminal paper\cite{lange1999reasons},
Lange and Oshima present at least seven reasons for using mobile agents in generic distributed systems.
In the context of WSNs, the most relevant ones are: network load reduction, network lantency
overcoming, protocol encapsulation, asynchronous and autonomous execution, robustness and fault-tolerance,
dynamic adaptation and orientation to heterogeneity.
 
\textit{Network load reduction.}  Mobile agents are lightweight modules and can be transfered between nodes to access remote resources
in order to save bandwidth resources. Also, a mobile agent with data processing capabilities can perform the needed operations on a
particular node and transmit the results to a sink. This avoids periodically sending raw sensed data from a node to the sink for computation
and data processing.
 
\textit{Network latency overcoming.} Agents can move to a remote node to perform required control tasks. This reduces the network
latency and will not affect the real-time control operations also in case of lack network connectivity with the base station.
 
\textit{Protocol encapsulation.} Mobile agents also facilitate easy modification application logic
 
\textit{Asynchronous and autonomous execution.} A mobile agent can traverse the network to gather the required information in order to
perform the required tasks and asynchronously send the result to the requester.
 
\textit{Dynamic adaptation.} Mobile agents can react autonomously to changes in the execution environment.
 
\textit{Robustness and fault-tolerance.} The mobile agents have the ability to dynamically react to events (e.g., low battery level) and can lead
to a robust and fault tolerant system (e.g., by migrating all executing agents to a similar node in order to continue their tasks).
 
\textit{Orientation to heterogeneity.} Mobile agents can offer a layer of abstraction among systems based on different hardware and software.
This ability enables the integration of heterogeneous WSNs supporting a variety of sensor platforms or connecting WSN and other networks
(like IP-based networks). However, the solutions presented in this survey do not facilitate this aspect. All have been designed for and tested
on a specific hardware/software platform.
 
In addition to \cite{aiello2008agents}, broadcasting lightweight modules consumes considerably less energy instead of the complete application\cite{cecilio2014existing}.
 
% Impala 2003
Impala\cite{liu2003impala} is a middleware solution that enables application modularity, adaptivity, and repairability in WSNs.
This system was part of the ZebraNet project, a mobile sensor network system for improving tracking technology
via energy-efficient tracking nodes and peer-to-peer communication techniques. Impala adopts OTAP (i.e., Over-The-Air Programming)
for updating the firmware on the nodes and describes a software architecture best suited for minimal performance and energy
impact for code running on resource constrained nodes. Mobility, openness, and scalability are supported by changing between
different protocols and modes of operation depending on the applications and network conditions. Impala is designed to execute
only on Hewlett-Packard/Compaq iPAQ Pocket PC handheld devices running Linux.
 
% Smart Messages 2004
Another modular based middleware is Smart Messages\cite{kang2004smartmessages} which proposes an autonomous
network architecture for large-scale embedded systems (NES). NES are known for restriction of resources, heterogeneity, and volatile nodes.
SMs aim to overcome these limitations by migrating the execution to nodes of interest, using application-controlled routing, instead of
end-to-end communication between nodes. The main contribution of this middleware is high-flexibility in the presence of dynamic network
configurations. SMs have also been designed and tested on Hewlett-Packard/Compaq iPAQ Pocket PC handheld devices running Linux.
 
% Agilla
Agilla\cite{fok2009agilla} is another example of this approach designed for WSNs and aims to reduce the code size by employing
a stack-based architecture. The purpose of this middleware is to support self-adaptiveness within the WSN by deploying up to four
autonomous mobile agents in each node. Agilla uses a \textit{tuple space} structure to ensure consistency in a dynamic environment
and to ensure scalability. A \textit{tuple space} is a type of shared memory in which data is structured as tuples and is accessed via
pattern-matching. In Agilla, the mobile agents are using \textit{tuple space} to discover properties of the environment.
To ensure scalability, application algorithms are designed to use local knowledge and interactions. Agilla runs only on top of TinyOS and
was deployed on Mica2, MicaZ, Tyndall 25mm and TelosB nodes.
 
% ActorNet
\textit{ActorNet}\cite{kwon2005actorne} is a mobile agent platform for WSNs designed to overcome the difficulties in allowing code
migration and interoperability due to the strict coupling between applications and sensor node architectures. ActorNet introduces
services like virtual memory, context switching and multi-tasking to enable the execution of complex, highly dynamic mobile
agent applications in the severely resource-constrained environment of WSNs. ActorNet has been specifically designed for TinyOS
on Mica2 sensor nodes.
 
% AFME
The AFME (Agent Factory Micro Edition)\cite{muldroon2006afme} middleware is a lightweight version of the Agent Factory framework
purposely designed for wireless pervasive systems to tackle the performance and memory footprint issues associated with executing agents
on mobile devices. It enables the creation of agent based applications on computationally constrained devices such as cellular digital mobile phones.
AFME was developed for J2ME enabled PDAs and then ported onto Sun SPOTs.
 
% MAPS
MAPS (Mobile Agent Platform for Sun SPOT)\cite{aiello2010maps} is a Java-based framework for WSNs based on Sun SPOT techology
which enables agent-oriented programming of WSN applications. MAPS offers a set of services to support mobile agents including
message transmission, agent creation, agent cloning, agent migration, time handling, and easy access to the sensor node resources.
The dynamic behaviour of agents is modeled as an ECA-based automaton. MAPS is also desgined to run only on Sun SPOTs.
 
% MASPOT
MASPOT\cite{lopes2011maspot} is another java-based mobile agent system for WSNs and builds on the concepts presented in
AFME and MAPS by providing code migration. This means that mobile agents can migrate to a node that is not yet configured. Additionally,
MASPOT also includes an agent management service.
 
% TinyMAPS
TinyMAPS\cite{aiello2012tinymaps} derives from MAPS and is designed for devices more constrained than the Sun SPOTs.
TinyMAPS can be considered as a porting of MAPS onto the Scentilla JCreate sensor platform. Scentilla sensors are much more
resource-constrained than Sun SPOT sensors.
 
%%%%%%%%%%%%%%%%%%%%%%%%%%%%%%%%%%%%%%%%%%%%%%
%%%%%%%%%%%%%%%%%%%%%%%%%%%%%%%%%%%%%%%%%%%%%%
 
 
 
 
%\newpage

\begin{table*}[h!]

\caption{Summary of the IoT middlewares in terms of supported
non-functional
 requirements}
\label{tab2}
 \centering

 {\scriptsize  \begin{tabular}{ |l|l|l|l|l|l|l|} \cline{1-7}
\multicolumn{7}{|c|}{Non-functional requirements}\\
\hline
& Scalability  & Security & Avaliability & Reliability & Real-time & Privacy \\
\hline
\multicolumn{7}{|c|}{Event-based}\\
%& Scalability  & Security & Avaliability & Reliability & Real-time & Privacy \\
\hline
Prisma & NL & NS&  NI& NI &  NRT& NI\\
\hline
Emma & NI & NI& NS & NS& HRT& NI\\
\hline
Hermes & AL& NI& S& NS& HRT& NI\\
\hline
Green & NL& NI& S& NS&HRT& NI\\
\hline
Mires & NL & NS & S & NS & NRT& NS \\
\hline
SensorBus &NI & S& S&NS &NRT& S \\
\hline

\multicolumn{7}{|c|}{Service-oriented}\\
 \hline
 Hydra & AL, NLWSNS & NS & NI & NI & SRT & NS\\
 \hline
SOCRADES~\cite{SOCRADES10} & AL, NLIoTS & C & NI & NI & SRT & NS\\
 \hline
 SenseWrap~\cite{SenseWrap09} & AL, NLIoTS & C & NI & NI & SRT & NS\\
 SENSEI~\cite{SENSEI10} & AL, NLWSNS & C & NI & NI & SRT & S\\
 \hline
UbiSOAP &N/A & Yes & Yes& Yes & No& N/A \\
\hline

 CarrIoT & Yes & N/A& Yes & Yes& Yes& N/A\\
 \hline
 SicsthSense & N/A& N/A& N/A & N/A &N/A & N/A\\
 \hline


\multicolumn{7}{|c|}{Component based approach}\\
\hline
\multicolumn{7}{|c|}{Agent-based approach}\\
\hline
ActorNet &N/A & N/A&N/A &N/A &N/A& N/A\\
\hline
Agilla & N/A & Yes& Yes& Yes& N/A& No\\
\hline
AFME &N/A & N/A& N/A& N/A& N/A& N/A \\
\hline
Impala &N/A &Yes &  Yes&  N/A& Yes& N/A \\
\hline
%& Scalability  & Security & Avaliability & Reliability & Real-time & Privacy \\
Maps &N/A & N/A& N/A& N/A& N/A& N/A \\
\hline
Maspot & N/A & N/A & N/A& N/A& No& N/A\\
\hline
Smart Messages & N/A & N/A & N/A& N/A& N/A& N/A\\
\hline
TinyMAPS & N/A & N/A & N/A& N/A& N/A& N/A\\
\hline
\multicolumn{7}{|c|}{Tuple space approach}\\
\hline
LIME &NI & NS& S&NS &HRT& S\\
\hline
UbiROAD &NI &S & S& NS&NRT& S\\
\hline
TeenyLIME & NI &NS & NS&NS &NRT &NS \\
\hline
TinyLime & NI & S & NS &NS &NRT& S\\
\hline
TS-Mid &NI &NS & NS&NS &NRT&NS \\
\hline
\multicolumn{7}{|c|}{Virtual machine approach}\\
 \hline
MAgNet & No& No& N/A &N/A &No& Yes\\
 \hline
Mate & No & Yes & N/A &N/A & No& N/A\\
 \hline
Muse & N/A& Yes&  N/A& N/A& No& No\\
 \hline
%& Scalability  & Security & Avaliability & Reliability & Real-time & Privacy \\
SwissQM &N/A &N/A &N/A &N/A &N/A& N/A\\
\hline
\multicolumn{7}{|c|}{Database approach}\\
\hline
SINA &NL &NS &NS &NS & NRT & NS\\
\hline
IrisNet & NI & NS & S& NS& HRT& S\\
\hline
HyCache & AL & NI &NS &S & NRT& NI \\
\hline
GSN &AL &S & S& NI& HRT&NS \\
\hline
KSpot\textsuperscript{+} & NL & S & S& NI& NRT& NS\\
\hline
Cougar & NL & S& S& NI & HRT&S \\
\hline
DsWare & NL & NS & NI & S & HRT& NS\\
\hline
Sensation & NL& NS &S & NI& HRT& NS\\
\hline
TinyDB & NI & NS  & NS& NS&  NRT & NS\\
\hline
\multicolumn{7}{|c|}{Application driven approach}\\
\hline
Milan & N/A & Yes & Yes& N/A& No& N/A\\
\hline
MidFusion & N/A & Yes &  Yes& N/A& Yes& No\\
\hline
%& Scalability  & Security & Avaliability & Reliability & Real-time & Privacy \\
Adaptive Middleware &N/A &N/A & N/A& N/A& No& N/A\\
\hline
AutoSec &Yes & N/A & Yes & Yes& Yes& N/A \\
\hline
\multirow{1}{*}{\textbf{Legend}} & Application Level (AL)  & Confidentiality(C)
& Supported (S) & Communication (CR) & Hard Real-Time (HRT) & Supported (S)\\
%\hline
& Network level (NL) & Integrity (I) &  &
  & Soft Real-Time (SRT) & \\
%\hline
 Not Supported (NS)  &\hspace{1mm}-IoT Scale(IoTS) & Availability (A) & & Data
 (DR) & Non Real-Time (NRT)  &  \\
No information (NI)  &\hspace{1mm}-WSN Scale(WSNS) & & & &   &  \\

\hline
\end{tabular} } \end{table*}


%\newpage

\begin{table*}[h!]
\caption{Summary of the IoT middlewares in terms of supported
architectural
 requirements}
\label{tab3}
 \centering
{\scriptsize  \begin{tabular}{ |l|l|l|l|l|l|l|l|l|} \cline{1-9}
\multicolumn{9}{|c|}{Architectural requirements}\\
\hline
& Abstraction & Interoperable & Context-aware & Autonomous & Adaptive & Service-Oriented & Lightweight & Distributed \\
\hline
\multicolumn{9}{|c|}{Event-based}\\
\hline
Prisma & REST & NL,AL & Monitor & NS & NS & Yes & Yes & Yes\\
\hline
Emma &NI & NS &NS & NS& Yes& Yes& Yes& Yes\\
\hline
Hermes & Yes& NS& NS & NS & NS& Yes& No& Yes\\
\hline
Green & XML& NS& Filter& NS & NS& Yes& No& Yes\\
\hline
Mires & NesC& NS& NS & NS & NS& Yes& NS& Yes\\
\hline
SensorBus & Yes&No &Yes & No&No &Yes & No & Yes\\
\hline


\multicolumn{9}{|c|}{Service-oriented}\\
\hline
Hydra & S & NI,SI,SeI & Y & NeI & DA & Y & E & Y\\
\hline
SOCRADES~\cite{SOCRADES10} & S & NeI & Y & Y & DA & Y & NI & Y\\
\hline
SenseWrap~\cite{SenseWrap09}& S & NeI & NI & NI & SA & Y & M & Y\\
\hline
SENSEI~\cite{SENSEI10}& S & NeI, SeI & Y & NI & DA & Y & NI & Y\\
\hline
CarrIoT & Yes & Yes & N/A& Yes& Yes& Yes& N/A& N/A\\
\hline
SicsthSense & Yes & N/A & N/A & N/A&N/A & N/A& N/A& N/A\\
\hline
UbiSOAP & N/A & Yes & Yes & Yes & No& Yes& Yes& Yes\\
\hline



\multicolumn{9}{|c|}{Component based}\\
\hline
\multicolumn{9}{|c|}{Agent-based approach}\\
\hline
ActorNet &N/A & N/A& N/A& N/A& N/A& N/A&N/A &N/A\\
\hline
Agilla &Yes &No & Yes& Yes& Yes& N/A& N/A& N/A\\
\hline
AFME &N/A & N/A& N/A& N/A& N/A& N/A&N/A &N/A\\
\hline
Impala & Yes &Yes & N/A& Yes&  Yes& N/A & Yes & N/A\\
\hline
Maps &N/A & N/A& N/A& N/A& N/A& N/A&N/A &N/A\\
%& Programming Abstraction & Interoperable & Context-aware & Autonomous & Adaptive & Service-Oriented & Lightweight & Distributed \\
\hline
Maspot &Yes & N/A& N/A & N/A& N/A & Yes & N/A &Yes\\
\hline
Smart Messages &N/A & N/A& N/A& N/A& N/A& N/A&N/A &N/A\\
\hline
TinyMAPS &N/A & N/A& N/A& N/A& N/A& N/A&N/A &N/A\\
\hline
\multicolumn{9}{|c|}{Tuple space approach}\\
\hline
LIME &GVDS &Yes &Yes &NS & NS& NI& Yes& Yes\\
\hline
UbiROAD & NI&Yes &Yes &Yes &NI &Yes & No&Yes\\
\hline
TeenyLIME &Yes & NI& NI& NI& NI& NS& Yes& Yes\\
\hline
TinyLime &Yes & NS& Yes& Yes& NS &NI & NI&Yes\\
\hline
TS-Mid &Yes &NS &NS &NS & NS& NS& No &Yes\\
\hline
\multicolumn{9}{|c|}{Virtual machine approach}\\
\hline
%& Programming Abstraction & Interoperable & Context-aware & Autonomous & Adaptive & Service-Oriented & Lightweight & Distributed \\
MAgNet &No & No& N/A& Yes& No & Yes& No& Yes\\
 \hline
Mate & Yes &N/A &N/A &N/A & Yes &N/A &N/A &N/A\\
 \hline
Muse &No  &No &No & No& No& Yes& Yes&Yes\\
 \hline
SwissQM & Yes & N/A& N/A& N/A& N/A&N/A & N/A&N/A\\
 \hline
\multicolumn{9}{|c|}{Database approach}\\
%& Programming Abstraction & Interoperable & Context-aware & Autonomous & Adaptive & Service-Oriented & Lightweight & Distributed \\
\hline
SINA & Yes& No& No& No& Yes& No&No &Yes\\
\hline
IrisNet &N/A & N/A& N/A& N/A& N/A& N/A&N/A &N/A\\
\hline
HyCache & N/A & N/A & N/A & N/A & N/A & No & No & Yes\\
\hline
GSN &Yes & N/A &N/A &N/A &Yes &Yes &N/A &N/A\\
\hline
KSpot\textsuperscript{+} &N/A & N/A& N/A& N/A& N/A& N/A&N/A &N/A\\
\hline
Cougar &N/A & N/A& N/A& N/A& N/A& N/A&N/A &N/A\\
\hline
DsWare & N/A &N/A & Yes& N/A& N/A& Yes& Yes& Yes\\
\hline
Sensation & Yes& Yes& Yes& Yes& No& N/A& N/A& Yes\\
\hline
TinyDB & Yes & N/A & N/A& No&  No&N/A & N/A & N/A\\
\hline
\multicolumn{9}{|c|}{Application driven approach}\\
\hline
Milan & Yes & Yes & Yes & No& No &Yes & No& Yes\\
%& Programming Abstraction & Interoperable & Context-aware & Autonomous & Adaptive & Service-Oriented & Lightweight & Distributed \\
\hline
MidFusion & Yes & No & Yes& Yes& Yes & No &Yes &\\
\hline
Adaptive Middleware & No &No & No& No& Yes & No& No &Yes\\
\hline
AutoSec & N/A & No& Yes& Yes& Yes& No& Yes&\\
\hline

\multirow{1}{*}{\textbf{Legend}} & Supported (S)  & Network (NeI) &
Yes (Y) & Yes (Y) & Dynamically (DA) & Yes (Y)& Energy (E) & Yes (Y)\\

  Not Supported (NS) & Not Supported (NS) & Syntactic (SI) & Not (N) & Not (N) &
  Statically (SA) & Not (N) & Memory(M) & Not (N)\\
%\hline
 No Information (NI) &  & Semantic (SeI) &  &  & &  &  & \\

\hline

\end{tabular} } \end{table*}

%\newpage
 
 \begin{table*}[h!]
\caption{Summary of the IoT groups of middleware in terms of
supported architectural
 requirements}
\label{tab3}
 \centering
{\tiny \begin{tabular}{ |l|l|l|l|l|l|l|l|l|} \cline{1-9}
\multicolumn{9}{|c|}{Architectural requirements}\\
\hline
& Programming Abstraction & Interoperable & Context-aware & Autonomous & Adaptive & Service-Oriented & Lightweight & Distributed \\
\hline Event-based & Partially& Partially  & Partially & Mostly No & Partially & Mostly Yes&Mostly Yes &Mostly Yes \\
\hline Service-oriented & Mostly Yes & Partially & Partially & Partially & Mostly Yes & Mostly Yes& Partially&Partially\\
\hline Component-based & Mostly Yes & Partially & Partially & Partially & Mostly Yes & Mostly Yes& Partially&Partially\\
\hline Agent-based & Partially & Partially & N/A & Partially & Partially & N/A & N/A &N/A\\
\hline Tuple space approach & Mostly Yes & Partially & Partially & Partially & Mostly No & Partially&Partially &Mostly Yes\\
\hline Virtual machine approach & Partially & Mostly No & Mostly No& Partially & Partially & Partially& Partially&Partially\\
\hline Data-based approach & Partially & Partially & Partially & Partially & Partially & Partially&Partially &Partially\\
\hline Application driven approach & Partially & Mostly No & Mostly Yes & Partially & Mostly Yes & Mostly No& Partially & Mostly Yes\\
\hline
\end{tabular} } \end{table*}

 
\section{Challenges and Open Issues}
 
\subsection{Challenges in designing middleware}
\begin{enumerate}
\item Limited power and resources
\begin{enumerate}
    \item Advance microelectronics technology allows tiny devices but limited in energy and resources, i.e. CPU and memory.
    \item Middleware should provide mechanisms for efficient processor and memory use while enabling lower-power communication and prolong sensor node lifetime. i.e. sleep mode, minimize number of transmission.
\end{enumerate}
 
\item Scalability, mobility, and dynamic network topology
\begin{enumerate}
    \item As the application grows, device failure, moving obstacles, mobility, and interference, the network will change frequently.
    \item Middleware should maintain performance and robust operation while network changes dynamically. Also, It should support mechanisms for fault tolerance and sensor node self-configuration and self-maintenance.
\end{enumerate}
 
\item Heterogeneity
\begin{enumerate}
    \item CPU-power, networking, memory and storage, operating systems.
    \item Middleware should able to interface various kinds of hardware, software and networks.
\end{enumerate}
 
\item Dynamic network organization
\begin{enumerate}
    \item In most situation, Client/Server mode is impossible, no infrastructure is not available.
    \item Middleware should support Ad-hoc capability to discover resource and its location which affect the trade-offs among latency, reliability, and energy.
\end{enumerate}
 
\item Real-world integration
\begin{enumerate}
    \item Most of the applications are dealing with real-time phenomena.
    \item Middleware should provide real-time services.
\end{enumerate}
 
\item Application knowledge
\begin{enumerate}
    \item Developer would like to inject application knowledge to the network so as to map application communication requirements to it,
    \item Middleware design should balance between application specificity and middleware generality.
\end{enumerate}
 
\item Data aggregation
\begin{enumerate}
    \item Network generate lots of redundant data, communications cost is much higher than computational cost. For example, sending a single bit can consume the same energy as executing 1000 instructions
    \item Middleware should able to aggregate data to eliminate redundancy and minimize the number of transmissions to the sink.
\end{enumerate}
 
\item Security
\begin{enumerate}
\item Middleware efforts should concentrate on developing and integrating security in the initial phases of software design, hence achieving different security requirements such as authentication, integrity, freshness, and availability.
\end{enumerate}
\end{enumerate}
 
 
 
 
\begin{itemize}
\item Privacy, Identity Management, Security and Access control: IoT presents significant challenges in terms of who can see what with which credentials
(recalling that the entities are no longer only people, but might be any form of IoT ``appliance''). The recent Stuxnet worm presents an excellent example of a malicious
"software actor" that has the potential to effect major physical changes in industrial processes. How does one guard against this type phenomenon in the IoT world?
\item Standardization and Interoperability: How do we make sure that the hugely diverse technology platform continues to act in a platform manner i.e. ensuring
that we do not have to re-invent the wheel every time we develop a new application or, indeed, a sensor that needs to plug into the IoT.
\item Data deluge: The IoT shares many of the key challenges similar to large scale data initiatives as identified in the e-Infrastructure domain.
How do we deal with the data stream of billions of "actors"? How do we ensure the data remains usable for future generations?
\item Scalability: Management of a network consists of trillion heterogeneous devices will be a serious challenge.
\end{itemize}
 


%% End of v2-acmsmall-sample.tex (March 2012) - Gerry Murray, ACM


%----------------------------------------------------------
% Style and layout of the references
%\bibliographystyle{elsarticle-num}
\bibliographystyle{IEEEtran}
\bibliography{MDRef} % Bibliography file (usually '*.bib' )



% that's all folks
\end{document}

















